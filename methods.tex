\chapter{Classes and Methods}
\label{Chp:Methods}

This chapter defines the behavior of all the methods in the GraphBLAS C API.
All methods can be declared for use in programs by including the {\tt GraphBLAS.h} header file.

We would like to emphasize that no GraphBLAS method will imply a predefined order over any associative operators. Implementations of the GraphBLAS are encouraged to exploit associativity to optimize performance of any GraphBLAS method. This holds even if the definition of the GraphBLAS method implies a fixed order for the associative operations.

%-----------------------------------------------------------------------------
\section{Context Methods}

%-----------------------------------------------------------------------------
\section{Object Classes and Methods}

This section describes methods that setup and operate on GraphBLAS opaque objects
but are not part of the the GraphBLAS math specification.  \scott{Does this need to be said?  Is it accurate?}

%-----------------------------------------------------------------------------
\subsection{Algebra Classes}

\scott{There are no separate "Types".  UnaryOp and BinaryOps are all callable objects.  Monoids and Semirings are class templates}

We could further define the concept and give a few examples of functions, functors and lambdas  that can be used where a unaryop, binaryop, monoid and semiring are needed.

Do we provide and list all of the predefined structs?


%-----------------------------------------------------------------------------
\subsection{Vector Class}

This section defines the minimum required public interface to the "API" (I refer to it
as the frontend) Vector class.  \scott{Are implementers free to add to the public API?
I think it is possible it may be necessary?}

\subsubsection{Template and Type Definitions}

No subclassing in API

\begin{verbatim}
    template<typename ScalarT, typename IndexT, typename... TagsT>
    class Vector
    {
    public:
        typedef ScalarT ScalarType;
        typedef IndexT  IndexType;
        
        // TODO: I think this can move to private section and be removed from spec:
        // implementation detail... implementations can add to this minimum specification
        //typedef typename detail::vector_generator::result<
        //    ScalarT,
        //    IndexT,
        //    detail::SparsenessCategoryTag,
        //    TagsT... ,
        //    detail::NullTag >::type BackendType;

        ...
    };
\end{verbatim}

We need to defined the template parameters:
\begin{itemize}
\item ScalarT
\item IndexT
\item TagsT variadic arguments
\end{itemize}

What are the possible types sent through TagsT?

The class needs to define a few typedefs
\begin{itemize}
\item ScalarType
\item IndexType
\end{itemize}

\subsubsection{Constructors and Destructors, Copies and Moves}

\begin{verbatim}
    // construction
    Vector() = delete;
    Vector(IndexType nsize);
    ~Vector();   // not virtual
    
    // copies (dup functionality)
    // moves (do we require move semantics?)
\end{verbatim}

\scott{class is a base class}
\scott{default constructor deleted}
\scott{destructor not virtual, do we allow implicit destructor?}
\scott{"dup" functionality: copy ctor and assignment operator both? What form?}
\scott{Do we require move semantics}


\subsubsection{{\sf build} Methods}

How many different overloads should be supported.

\begin{verbatim}
    // build methods
    template<typename RAIteratorI,
             typename RAIteratorV,
             typename BinaryOpT = GraphBLAS::Second<ScalarType> >
    void build(RAIteratorI  i_it,
               RAIteratorV  v_it,
               IndexType    num_vals,
               BinaryOpT    dup = BinaryOpT());

    template<typename BinaryOpT = GraphBLAS::Second<ScalarType> >
    inline void build(std::vector<IndexType>     const &indices,
                      std::vector<ScalarType>    const &values,
                      BinaryOpT                   dup = BinaryOpT());
\end{verbatim}

\subsubsection{{\sf clear} Method}
\begin{verbatim}
    void clear();
\end{verbatim}

\subsubsection{{\sf size} Method}
\begin{verbatim}
    IndexType size() const;
\end{verbatim}

\subsubsection{{\sf nvals} Method}
\begin{verbatim}
    IndexType nvals() const;
\end{verbatim}

\subsubsection{{\sf resize} Method}
\begin{verbatim}
    void resize(IndexType new_size);
\end{verbatim}


\subsubsection{{\sf hasElement} Method}
\begin{verbatim}
    bool hasElement(IndexType index) const;
\end{verbatim}

\subsubsection{{\sf setElement} Method}
\begin{verbatim}
    void setElement(IndexType index, ScalarT const &new_val);
\end{verbatim}

\subsubsection{{\sf removeElement} Method}
\begin{verbatim}
    void removeElement(IndexType index);
\end{verbatim}

\subsubsection{{\sf extractElement} Method}
\begin{verbatim}
    ScalarT extractElement(IndexType index) const;
\end{verbatim}


\subsubsection{{\sf extractTuples} Method}
\begin{verbatim}
    template<typename RAIteratorIT,
             typename RAIteratorVT>
    void extractTuples(RAIteratorIT        i_it,
                       RAIteratorVT        v_it) const;

    void extractTuples(IndexArrayType        &indices,
                       std::vector<ScalarT>  &values) const;
\end{verbatim}


%-----------------------------------------------------------------------------
\subsection{Matrix Class}

This section defines the minimum required public interface to the "API" (I refer to it
as the frontend) Matrix class.  \scott{Are implementers free to add to the public API?
I think it is possible it may be necessary?}

%-----------------------------------------------------------------------------
\subsection{Descriptor Class}

\scott{There should be no Descriptors...I hope...maybe Views}.

%-----------------------------------------------------------------------------
%-----------------------------------------------------------------------------
\subsection{Views}

\subsubsection{transpose method, TransposeView class}

Only for matrices.  Do we need to specify the TransposeView class?
Is the class an implementation detail?  Can we specify the method without stating its return type?


\subsubsection{structure method, StructureView class}

Only for masks.  Do we need to specify the StructureView class?
Is the class an implementation detail?  Can we specify the method without stating its return type?


\subsubsection{complement method, ComplementView class}

Only for masks.  Do we need to specify the ComplementView class?
Is the class an implementation detail?  Can we specify the method without stating its return type?


\subsubsection{Composing mask views}

Only complement(structure(mask)) is valid  (structure(complement(mask) is not valid).

Do we talk about StructuralComplementView class?

Is the class an implementation detail?  Can we specify the method without stating its return type?

%-----------------------------------------------------------------------------
\subsection{{\sf free} Methods}

\scott{should be replaced with class destructors}.

%-----------------------------------------------------------------------------
\section{GraphBLAS Operations}

\subsection{ops\_mxm\_vxm\_mxv}
\subsection{ops\_ewisemult\_ewiseadd}
\subsection{ops\_extract}
\subsection{ops\_assign} 
\subsection{ops\_apply}
\subsection{ops\_reduce\_transpose}
\subsection{ops\_kronecker}

%-----------------------------------------------------------------------------
\section{Sequence Termination Methods}