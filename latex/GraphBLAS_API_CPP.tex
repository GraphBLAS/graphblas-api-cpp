\documentclass[11pt]{extbook}

\usepackage{geometry}

\usepackage{fancyvrb}
\usepackage{color}
\usepackage{graphicx}
\usepackage{fullpage}
\usepackage{verbatim}
\usepackage{tikz}
\usepackage{listings}
\usepackage[yyyymmdd,hhmmss]{datetime}
\usepackage{rotating}
\usepackage{authblk}
\usepackage{amsfonts}
\usepackage{amsmath}
\usepackage{amssymb}
\usepackage{todonotes}
\usepackage{titlesec}
\usepackage[mathlines]{lineno}
\usepackage{tabularx}
\usepackage{enumitem}
\usepackage{hyperref}
\usepackage{bm}
\usepackage{etoolbox}
\usepackage{pdflscape}
\usepackage{threeparttable}

%TGM:  Added these packages to fix underscore rendering
\usepackage{lmodern} 
\usepackage[T1]{fontenc}

\setcounter{secnumdepth}{3}
\setcounter{tocdepth}{3}

%SM:  Comment these to remove DRAFT watermark
\usepackage[color={[gray]{0.93}}]{draftwatermark}
\SetWatermarkText{DRAFT}
\SetWatermarkScale{0.9}

\renewcommand{\thefootnote}{\fnsymbol{footnote}}
\setcounter{footnote}{1}

\titleformat{\paragraph}
{\normalfont\normalsize\bfseries}{\theparagraph}{1em}{}
\titlespacing*{\paragraph}
{0pt}{3.25ex plus 1ex minus .2ex}{1.5ex plus .2ex}

\newtoggle{assign}
\toggletrue{assign}

\newcommand{\qg}{\u{g}}
\newcommand{\qG}{\u{G}}
\newcommand{\qc}{\c{c} }
\newcommand{\qC}{\c{C}}
\newcommand{\qs}{\c{s}}
\newcommand{\qS}{\c{S}}
\newcommand{\qu}{\"{u}}
\newcommand{\qU}{\"{U}}
\newcommand{\qo}{\"{o}}
\newcommand{\qO}{\"{O}}
\newcommand{\qI}{\.{I}}
\newcommand{\wa}{\^{a}}
\newcommand{\wA}{\^{A}}

\begin{document}

\linenumbers

\title{
The GraphBLAS C++ API Specification
\footnote{Based on \emph{GraphBLAS Mathematics} by Jeremy Kepner}: \\ 
{\large Version 1.x.x} \\
{\normalsize \scott{Remember to update acknowledgements and remove DRAFT}}
}

\author{Benjamin Brock, Ayd\i n Bulu\c{c}, Timothy Mattson, Scott McMillan, Jos\'e Moreira}

\date{Generated on \today\ at \currenttime\ EDT}

\newcommand{\kron}{\mathbin{\text{\footnotesize \textcircled{\raisebox{-0.3pt}{\footnotesize $\otimes$}}}}}
\newcommand{\grbarray}[1]{\bm{#1}}
\renewcommand{\vector}[1]{{\bf #1}}
\renewcommand{\matrix}[1]{{\bf #1}}
\renewcommand{\arg}[1]{{\sf #1}}
\newcommand{\zip}{{\mbox{zip}}}
\newcommand{\zap}{{\mbox{zap}}}
\newcommand{\ewiseadd}{{\mbox{\bf ewiseadd}}}
\newcommand{\ewisemult}{{\mbox{\bf ewisemult}}}
\newcommand{\mxm}{{\mbox{\bf mxm}}}
\newcommand{\vxm}{{\mbox{\bf vxm}}}
\newcommand{\mxv}{{\mbox{\bf mxv}}}
\newcommand{\gpit}[1]{{\sf #1}}
\newcommand{\ie}{{i.e.}}
\newcommand{\eg}{{e.g.}}
\newcommand{\nan}{{\sf NaN}}
\newcommand{\nil}{{\bf nil}}
\newcommand{\ifif}{{\bf if}}
\newcommand{\ifthen}{{\bf then}}
\newcommand{\ifelse}{{\bf else}}
\newcommand{\ifendif}{{\bf endif}}
\newcommand{\zero}{{\bf 0}}
\newcommand{\one}{{\bf 1}}
\newcommand{\true}{{\sf true}}
\newcommand{\false}{{\sf false}}
\newcommand{\syntax}{{C++ Syntax}}

\newcommand{\Dinn}{\mbox{$D_{in}$}}
\newcommand{\Din}[1]{\mbox{$D_{in_{#1}}$}}
\newcommand{\Dout}{\mbox{$D_{out}$}}

\newcommand{\bDinn}{\mbox{$\mathbf{D}_{in}$}}
\newcommand{\bDin}[1]{\mbox{$\mathbf{D}_{in_{#1}}$}}
\newcommand{\bDout}{\mbox{$\mathbf{D}_{out}$}}

\newcommand{\aydin}[1]{{{\color{orange}[Aydin: #1]}}}
\newcommand{\scott}[1]{{{\color{violet}[Scott: #1]}}}
\newcommand{\tim}[1]{{{\color{teal}[Tim: #1]}}}
\newcommand{\jose}[1]{{{\color{red}[Jose: #1]}}}
\newcommand{\ben}[1]{{{\color{blue}[Ben: #1]}}}

%\newcommand{\aydin}[1]{}
%\newcommand{\scott}[1]{}
%\newcommand{\tim}[1]{}
%\newcommand{\jose}[1]{}
%\newcommand{\ben}[1]{}

%\carl{testing}
%\scott{testing}
%\aydin{testing}
%\tim{testing}
%\jose{testing}
%\ajy{testing}

\renewcommand{\comment}[1]{{}}
\newcommand{\glossBegin}{\begin{itemize}}
\newcommand{\glossItem}[1]{\item \emph{#1}: }
\newcommand{\glossEnd}{\end{itemize}}

\setlength{\parskip}{0.5\baselineskip}
\setlength{\parindent}{0ex}

\maketitle


\renewcommand{\thefootnote}{\arabic{footnote}}
\setcounter{footnote}{0}

\vfill

Copyright \copyright\ 2020 Carnegie Mellon University, The Regents 
of the University of California, through Lawrence Berkeley National 
Laboratory (subject to receipt of any required approvals from the 
U.S. Dept. of Energy), the Regents of the University of California 
(U.C. Berkeley), Intel Corporation, International Business Machines 
Corporation, and Massachusetts Institute of Technology Lincoln
Laboratory. 

Any opinions, findings and conclusions or recommendations expressed in 
this material are those of the author(s) and do not necessarily reflect 
the views of the United States Department of Defense, the United States 
Department of Energy, Carnegie Mellon University, the Regents of the 
University of California, Intel Corporation, or the IBM Corporation.  

NO WARRANTY. THIS MATERIAL IS FURNISHED ON AN AS-IS BASIS. THE COPYRIGHT 
OWNERS AND/OR AUTHORS MAKE NO WARRANTIES OF ANY KIND, EITHER EXPRESSED 
OR IMPLIED, AS TO ANY MATTER INCLUDING, BUT NOT LIMITED TO, WARRANTY OF 
FITNESS FOR PURPOSE OR MERCHANTABILITY, EXCLUSIVITY, OR RESULTS OBTAINED 
FROM USE OF THE MATERIAL. THE COPYRIGHT OWNERS AND/OR AUTHORS DO NOT MAKE 
ANY WARRANTY OF ANY KIND WITH RESPECT TO FREEDOM FROM PATENT, TRADE MARK, 
OR COPYRIGHT INFRINGEMENT.

\vspace{1.5cm}

\vspace{2cm}
%{\Large This version is a definitive release of the GraphBLAS C API
%specification. As of the date of this document, at least two independent
%and functionally complete implementations are available.}

{\Large This version is a provisional release of the GraphBLAS C++ API specification.
Once two functionally complete reference implementations are available, we
will remove the "provisional" clause.}


\vspace{1.5cm}


%[Distribution Statement A] This material has been approved for public release and unlimited distribution.  
%Please see Copyright notice for non-US Government use and distribution.

Except as otherwise noted, this material is licensed under a Creative Commons Attribution 4.0 license (\href{http://creativecommons.org/licenses/by/4.0/legalcode}{http://creativecommons.org/licenses/by/4.0/legalcode}), 
and examples are licensed under the BSD License (\href{https://opensource.org/licenses/BSD-3-Clause}{https://opensource.org/licenses/BSD-3-Clause}).

%\begin{abstract}
%\end{abstract}

\vfill
\pagebreak

%-----------------------------------------------------------------------------
\tableofcontents
\vfill
\pagebreak

%-----------------------------------------------------------------------------
\addcontentsline{toc}{section}{List of Tables}
\listoftables
\vfill
\pagebreak

\addcontentsline{toc}{section}{List of Figures}
\listoffigures
\vfill
\pagebreak

%-----------------------------------------------------------------------------

\section*{Acknowledgments}
\addcontentsline{toc}{section}{Acknowledgments}

This document represents the work of the people who have served on the C++ API
Subcommittee of the GraphBLAS Forum.

Those who served as C++ API Subcommittee members for GraphBLAS 1.x.x are (in alphabetical order):
\begin{itemize}
\item Benjamin Brock (UC Berkeley)
\item Ayd\i n Bulu\c{c} (Lawrence Berkeley National Laboratory)
\item Timothy G. Mattson (Intel Corporation)
\item Scott McMillan (Software Engineering Institute at Carnegie Mellon University)
\item Jos\'e Moreira (IBM Corporation)
\end{itemize}

The GraphBLAS specification is based upon work funded and supported in part by:
\begin{itemize}
\item The Department of Energy Office of Advanced Scientific Computing Research under contract number DE-AC02-05CH11231
\item Intel Corporation
\item Department of Defense under Contract No. FA8702-15-D-0002 with Carnegie Mellon University for the operation of the Software Engineering Institute [DM-0003727, DM19-0929]
\item International Business Machines Corporation
\item Department of Defense under contract No. W911QX-12-C-0059, L-3 Data Tactics subcontract SCT-14-004 with University of California, Davis
\item NSF Graduate Research Fellowship under Grant No. DGE 1752814 and by the NSF under Award No. 1823034 with the University of California, Berkeley
\end{itemize}

The following people provided valuable input and feedback during the development of the specification (in alphabetical order):
Hollen Barmer, Tim Davis, Jeremy Kepner, Peter Kogge, Manoj Kumar, Andrew Mellinger, 
Maxim Naumov, Nancy M. Ott, Ping Tak Peter Tang, Michael Wolf, Carl Yang, Albert-Jan Yzelman.
\vfill
\pagebreak

%-----------------------------------------------------------------------------

\chapter{Introduction}

The GraphBLAS standard defines a set of matrix and vector operations 
based on semi-ring algebraic structures.  
These operations can be used
to express a wide range of graph algorithms.   This document 
defines the C++ binding to the GraphBLAS standard.   We refer to 
this as the {\it GraphBLAS C++ API} (Application Programming Interface).   

The GraphBLAS C++ API is built on a collection of   
objects exposed to the C++ programmer as opaque data types. 
Functions that manipulate these
objects are referred to as {\it methods}.  These methods fully define the 
interface to GraphBLAS objects to create or destroy them, modify their 
contents, and copy the contents of opaque objects into non-opaque objects; the 
contents of which are under direct control of the programmer.
\scott{We need to adopt C++ terminology, class methods or member functions, and free functions for operations}.

The GraphBLAS C++ API is designed to work with C++17/14/11 (ISO/IEC xxxx:xxx) 
\scott{need to figure out which version of the language we are going to require}
extended with {\it static type-based} and {\it number of parameters-based}.  
Furthermore, the standard assumes programs using the GraphBLAS C++ API
will execute on hardware that supports floating point arithmetic
such as that defined by the IEEE~754 (IEEE 754-2008) standard. 

The remainder of this document is organized as follows:
\begin{itemize}
\item Chapter~\ref{Chp:Concepts}: Basic Concepts
\item Chapter~\ref{Chp:Objects}: Objects
\item Chapter~\ref{Chp:Methods}: Methods
\item Appendix~\ref{Chp:RevHistory}: Revision History
\item Appendix~\ref{Chp:Examples}: Examples
\end{itemize}

%=============================================================================
%=============================================================================

\chapter{Basic Concepts}
\label{Chp:Concepts}

% ========================================================================
\section{Glossary}

\subsection{GraphBLAS API basic definitions}

\subsection{GraphBLAS objects and their structure}

\subsection{Algebraic structures used in the GraphBLAS}

Operators must comply with the style dictated for callables from {\sf <functional>} 
header file and lambdas.  Recommend C++14 approach that has deprecated use of {\sf result\_type}
typedefs and instead use
\begin{verbatim}
   decltype(binaryop(std::declval<typename LHS::ScalarType>(),
                     std::declval<typename RHS::ScalarType>()))
\end{verbatim}


GBTL demostrates how lambas and things like std::bind objects can be passed 
as arguments in place of UnaryOp and BinaryOp (they follow the Callable 
concept).  

Monoids and Semirings do not follow the Callable Concept.  These are
structs with methods not restricted to operator()().


\subsection{The execution of an application using the GraphBLAS C API}

\subsection{GraphBLAS methods: behaviors and error conditions}


% ========================================================================
\section{Notation}


% ========================================================================
\section{Algebraic and Arithmetic Foundations}


% ========================================================================
\section{GraphBLAS Opaque Objects}

Not GrB\_Type? (should not exist)\\
Not GrB\_Descriptor (does not exist)

While we have the concepts of UnaryOp, BinaryOp, Monoid, and Semiring we 
do not need to have such classes.  GBTL demostrates how lambas and things 
like std::bind objects can be passed as arguments in place of UnaryOp and
BinaryOp (they follow the Callable concept).  Monoids and Semirings do not 
follow the Callable Concept.

Right now only Vector and Matrix for sure.

There is a frontend Matrix and Vector class that take the place of C handles.

% ========================================================================
\section{Domains}


% ========================================================================
\section{Operators and Associated Functions}

While we have the concepts of UnaryOp, BinaryOp, Monoid, and Semiring we do not have such classes

See GBTL's {\sf algrebra.hpp} for demonstration of all "predefined" operators.


% ========================================================================
\section{Indices, Index Arrays, and Scalar Arrays}

Unlike the C API, GraphBLAS Vectors and Matrices in C++ can be templatized on
the index type and so can take on any unsigned integer type; therefore, there is
no equivalent to GrB\_Index.

Need to deal with mismatched index types?

% ========================================================================
\section{Execution Model}

\subsection{Execution modes}

\subsection{Thread safety}
\label{Sec:ThreadSafety}


% ========================================================================
\section{Error Model}
\label{Sec:ErrorModel}

Depending on support for execution models we may or may not have two 
categories of errors: API and execution errors.

All runtime errors are reported by throwing exceptions.  All exceptions generated
by the library inherit from {\sf grb::exception} that (in turn) inherits 
from {\sf std::exception}.

Exceptions subclassed from {\sf grb::exception}:

\begin{itemize}
\item {\sf logic\_error} base class for API errors.
    \begin{itemize}
    \item {\sf dimension\_mismatch}
    \end{itemize}
\item {\sf runtime\_error} base class for execution errors.
    \begin{itemize}
    \item {\sf panic\_error}
    \item {\sf bad\_alloc}... should we use {\sf std::bad\_alloc} instead?
    \item {\sf invalid\_object}
    \end{itemize}
\end{itemize}

% ========================================================================
\section{Named Requirements}
This section contains \textit{named requirements} describing the requirements for
various types to be used by the C++ GraphBLAS standard.
GraphBLAS implementations may choose to formalize these named requirements with
C++ concepts.  Regardless, C++ GraphBLAS implementers are responsible for
ensuring that their implementations comply with these named requirements, and
users who pass in custom objects to the GraphBLAS API are responsible for
ensuring that their types satisfy the corresponding named requirements.

\subsection{Binary Operators}
An object whose type fulfills the \textit{BinaryOperator} requirement is a C++
\textit{Callable} that can be called with two arguments, returning a single
value.  We call the type of the lefthand argument passed in to the binary operator
\texttt{T} and the type of the righthand argument \texttt{U}.  A callable is a
valid binary operator for types \texttt{T} and \texttt{U} if it is callable with
arguments of those types, returning an object of type \texttt{V}.

\begin{tabularx}{\textwidth}{l l X}
\textbf{Expression} & \textbf{Return Type} & \textbf{Requirements}\\
\hline
\texttt{binary\_op(T(), U())} & \texttt{V} & \texttt{binary\_op} is callable \texttt{T} $\times$ \texttt{U} $\rightarrow$ \texttt{V}.\\
\end{tabularx}

\subsection{Monoids}
An object whose type fulfills the \textit{Monoid} requirement is a callable that
fulfills the \textit{BinaryOperator} requirement in addition to two other requirements.
We say that an object is a monoid on type \texttt{T} if:

\begin{enumerate}
   \item The monoid is a valid \textit{BinaryOperator} on type \texttt{T} and has a return value of type \texttt{T}, \texttt{T} $\times$ \texttt{T} $\rightarrow$ \texttt{T}.
   \item The monoid has a method \texttt{identity} that is callable with template parameter \texttt{T}, returning an object of type \texttt{T} which is the identity for the binary operator on that type.
\end{enumerate}

\begin{tabularx}{\textwidth}{l l X}
\textbf{Expression} & \textbf{Return Type} & \textbf{Requirements}\\
\hline
\texttt{monoid(T(), T())} & \texttt{T} & \texttt{monoid} is callable \texttt{T} $\times$ \texttt{T} $\rightarrow$ \texttt{T}.\\
\hline
\texttt{monoid.identity<T>()} & \texttt{T} & \texttt{identity()} method callable with template parameter \texttt{T}.\\
\end{tabularx}

\chapter{Objects}
\label{Chp:Objects}


% ========================================================================
\section{Operators}

\scott{All callable objects must be supported, including <functional> and lambdas.}

% ========================================================================
\section{Monoids}

\scott{Callables + identity()?}

% ========================================================================
\section{Semirings}

\scott{likely only structs}

% ========================================================================
\section{Vectors}
\label{Sec:Vectors}

\scott{Add a discussion of template parameters}

% ========================================================================
\section{Matrices}
\label{Sec:Matrices}

\scott{Add a discussion of template parameters}

% ========================================================================
\section{Masks}
\label{Sec:Masks}

% ========================================================================
\section{Descriptors}
\label{Sec:Descriptors}

\scott{There will be no descriptors...I hope}
\chapter{Classes and Methods}
\label{Chp:Methods}

This chapter defines the behavior of all the methods in the GraphBLAS C API.
All methods can be declared for use in programs by including the {\tt GraphBLAS.h} header file.

We would like to emphasize that no GraphBLAS method will imply a predefined order over any associative operators. Implementations of the GraphBLAS are encouraged to exploit associativity to optimize performance of any GraphBLAS method. This holds even if the definition of the GraphBLAS method implies a fixed order for the associative operations.

%-----------------------------------------------------------------------------
\section{Context Methods}

%-----------------------------------------------------------------------------
\section{Object Classes and Methods}

This section describes methods that setup and operate on GraphBLAS opaque objects
but are not part of the the GraphBLAS math specification.  \scott{Does this need to be said?  Is it accurate?}

%-----------------------------------------------------------------------------
\subsection{Algebra Classes}

\scott{There are no separate "Types".  UnaryOp and BinaryOps are all callable objects.  Monoids and Semirings are class templates}

We could further define the concept and give a few examples of functions, functors and lambdas  that can be used where a unaryop, binaryop, monoid and semiring are needed.

Do we provide and list all of the predefined structs?


%-----------------------------------------------------------------------------
\subsection{Vector Class}

This section defines the minimum required public interface to the "API" (I refer to it
as the frontend) Vector class.  \scott{Are implementers free to add to the public API?
I think it is possible it may be necessary?}

\subsubsection{Template and Type Definitions}

No subclassing in API

\begin{verbatim}
    template<typename ScalarT, typename IndexT, typename... TagsT>
    class Vector
    {
    public:
        typedef ScalarT ScalarType;
        typedef IndexT  IndexType;
        
        // TODO: I think this can move to private section and be removed from spec:
        // implementation detail... implementations can add to this minimum specification
        //typedef typename detail::vector_generator::result<
        //    ScalarT,
        //    IndexT,
        //    detail::SparsenessCategoryTag,
        //    TagsT... ,
        //    detail::NullTag >::type BackendType;

        ...
    };
\end{verbatim}

We need to defined the template parameters:
\begin{itemize}
\item ScalarT
\item IndexT
\item TagsT variadic arguments
\end{itemize}

What are the possible types sent through TagsT?

The class needs to define a few typedefs
\begin{itemize}
\item ScalarType
\item IndexType
\end{itemize}

\subsubsection{Constructors and Destructors, Copies and Moves}

\begin{verbatim}
    // construction
    Vector() = delete;
    Vector(IndexType nsize);
    ~Vector();   // not virtual
    
    // copies (dup functionality)
    // moves (do we require move semantics?)
\end{verbatim}

\scott{class is a base class}
\scott{default constructor deleted}
\scott{destructor not virtual, do we allow implicit destructor?}
\scott{"dup" functionality: copy ctor and assignment operator both? What form?}
\scott{Do we require move semantics}


\subsubsection{{\sf build} Methods}

How many different overloads should be supported.

\begin{verbatim}
    // build methods
    template<typename RAIteratorI,
             typename RAIteratorV,
             typename BinaryOpT = GraphBLAS::Second<ScalarType> >
    void build(RAIteratorI  i_it,
               RAIteratorV  v_it,
               IndexType    num_vals,
               BinaryOpT    dup = BinaryOpT());

    template<typename BinaryOpT = GraphBLAS::Second<ScalarType> >
    inline void build(std::vector<IndexType>     const &indices,
                      std::vector<ScalarType>    const &values,
                      BinaryOpT                   dup = BinaryOpT());
\end{verbatim}

\subsubsection{{\sf clear} Method}
\begin{verbatim}
    void clear();
\end{verbatim}

\subsubsection{{\sf size} Method}
\begin{verbatim}
    IndexType size() const;
\end{verbatim}

\subsubsection{{\sf nvals} Method}
\begin{verbatim}
    IndexType nvals() const;
\end{verbatim}

\subsubsection{{\sf resize} Method}
\begin{verbatim}
    void resize(IndexType new_size);
\end{verbatim}


\subsubsection{{\sf hasElement} Method}
\begin{verbatim}
    bool hasElement(IndexType index) const;
\end{verbatim}

\subsubsection{{\sf setElement} Method}
\begin{verbatim}
    void setElement(IndexType index, ScalarT const &new_val);
\end{verbatim}

\subsubsection{{\sf removeElement} Method}
\begin{verbatim}
    void removeElement(IndexType index);
\end{verbatim}

\subsubsection{{\sf extractElement} Method}
\begin{verbatim}
    ScalarT extractElement(IndexType index) const;
\end{verbatim}


\subsubsection{{\sf extractTuples} Method}
\begin{verbatim}
    template<typename RAIteratorIT,
             typename RAIteratorVT>
    void extractTuples(RAIteratorIT        i_it,
                       RAIteratorVT        v_it) const;

    void extractTuples(IndexArrayType        &indices,
                       std::vector<ScalarT>  &values) const;
\end{verbatim}


%-----------------------------------------------------------------------------
\subsection{Matrix Class}

This section defines the minimum required public interface to the "API" (I refer to it
as the frontend) Matrix class.  \scott{Are implementers free to add to the public API?
I think it is possible it may be necessary?}

%-----------------------------------------------------------------------------
\subsection{Descriptor Class}

\scott{There should be no Descriptors...I hope...maybe Views}.

%-----------------------------------------------------------------------------
%-----------------------------------------------------------------------------
\subsection{Views}

\subsubsection{transpose method, TransposeView class}

Only for matrices.  Do we need to specify the TransposeView class?
Is the class an implementation detail?  Can we specify the method without stating its return type?


\subsubsection{structure method, StructureView class}

Only for masks.  Do we need to specify the StructureView class?
Is the class an implementation detail?  Can we specify the method without stating its return type?


\subsubsection{complement method, ComplementView class}

Only for masks.  Do we need to specify the ComplementView class?
Is the class an implementation detail?  Can we specify the method without stating its return type?


\subsubsection{Composing mask views}

Only complement(structure(mask)) is valid  (structure(complement(mask) is not valid).

Do we talk about StructuralComplementView class?

Is the class an implementation detail?  Can we specify the method without stating its return type?

%-----------------------------------------------------------------------------
\subsection{{\sf free} Methods}

\scott{should be replaced with class destructors}.

%-----------------------------------------------------------------------------
\section{GraphBLAS Operations}

\subsection{ops\_mxm\_vxm\_mxv}
\subsection{ops\_ewisemult\_ewiseadd}
\subsection{ops\_extract}
\subsection{ops\_assign} 
\subsection{ops\_apply}
\subsection{ops\_reduce\_transpose}
\subsection{ops\_kronecker}

%-----------------------------------------------------------------------------
\section{Sequence Termination Methods}


%=============================================================================
%=============================================================================

\appendix
\chapter{Revision History}
\label{Chp:RevHistory}

Initial release 1.x.x


%--------------------------------------------------------------

\chapter{Examples}
\label{Chp:Examples}

\pagebreak
\nolinenumbers
\section{Example: level breadth-first search (BFS) in GraphBLAS}
{\scriptsize
%\lstinputlisting[language=C,numbers=left]{BFS5M.cpp}
}
\vfill

\pagebreak
\nolinenumbers
\section{Example: level BFS in GraphBLAS using apply}
{\scriptsize
%\lstinputlisting[language=C,numbers=left]{BFS6_apply.cpp}
}
\vfill

\pagebreak
\nolinenumbers
\section{Example: parent BFS in GraphBLAS}
{\scriptsize
%\lstinputlisting[language=C,numbers=left]{BFS7_parents.cpp}
}
\vfill

\pagebreak
\nolinenumbers
\section{Example: betweenness centrality (BC) in GraphBLAS}
\label{App:BCnobatch}
{\scriptsize
%\lstinputlisting[language=C,numbers=left]{BC1M_update.cpp}
}
\vfill

\pagebreak
\nolinenumbers
\section{Example: batched BC in GraphBLAS}
{\scriptsize
%\lstinputlisting[language=C,escapechar=|,numbers=left]{BC1_batch.cpp}
}
\vfill

\pagebreak
\nolinenumbers
\section{Example: maximal independent set (MIS) in GraphBLAS}
{\scriptsize
%\lstinputlisting[language=C,numbers=left]{MIS1.cpp}
}
\vfill

\pagebreak
\nolinenumbers
\section{Example: counting triangles in GraphBLAS}
{\scriptsize
%\lstinputlisting[language=C,numbers=left]{TC1.cpp}
}
\vfill
\pagebreak


%\def\IEEEbibitemsep{3pt plus .5pt}
%\bibliographystyle{IEEEtran}
%\bibliography{refs}

\end{document}
