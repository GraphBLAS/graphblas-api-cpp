\documentclass[11pt]{extbook}

\usepackage{geometry}
\usepackage{fancyvrb}
\usepackage{color}
\usepackage{graphicx}
\usepackage{fullpage}
\usepackage{verbatim}
\usepackage{tikz}
\usepackage{listings}
\usepackage[yyyymmdd,hhmmss]{datetime}
\usepackage{rotating}
\usepackage{authblk}
\usepackage{amsfonts}
\usepackage{amsmath}
\usepackage{amssymb}
\usepackage{todonotes}
\usepackage{titlesec}
\usepackage[mathlines]{lineno}
\usepackage{tabularx}
\usepackage{minted}
\usepackage{enumitem}
\usepackage{hyperref}
\usepackage{bm}
\usepackage{etoolbox}
\usepackage{pdflscape}
\usepackage{threeparttable}

%TGM:  Added these packages to fix underscore rendering
\usepackage{lmodern} 
\usepackage[T1]{fontenc}

\setcounter{secnumdepth}{3}
\setcounter{tocdepth}{3}

%SM:  Comment these to remove DRAFT watermark
\usepackage[color={[gray]{0.93}}]{draftwatermark}
%\usepackage{draftwatermark}
\SetWatermarkText{DRAFT}
\SetWatermarkScale{0.9}

\renewcommand{\thefootnote}{\fnsymbol{footnote}}
\setcounter{footnote}{1}

\titleformat{\paragraph}
{\normalfont\normalsize\bfseries}{\theparagraph}{1em}{}
\titlespacing*{\paragraph}
{0pt}{3.25ex plus 1ex minus .2ex}{1.5ex plus .2ex}

\newtoggle{assign}
\toggletrue{assign}

\newcommand{\qg}{\u{g}}
\newcommand{\qG}{\u{G}}
\newcommand{\qc}{\c{c} }
\newcommand{\qC}{\c{C}}
\newcommand{\qs}{\c{s}}
\newcommand{\qS}{\c{S}}
\newcommand{\qu}{\"{u}}
\newcommand{\qU}{\"{U}}
\newcommand{\qo}{\"{o}}
\newcommand{\qO}{\"{O}}
\newcommand{\qI}{\.{I}}
\newcommand{\wa}{\^{a}}
\newcommand{\wA}{\^{A}}

\begin{document}

\linenumbers

\title{
The GraphBLAS C++ API Specification
\footnote{Based on \emph{GraphBLAS Mathematics} by Jeremy Kepner}: \\ 
{\large Version 1.0.0} \\
{\normalsize \scott{Remember to update acknowledgements and remove DRAFT}}
}

\author{Benjamin Brock and Scott McMillan, \\ with Ayd\i n Bulu\c{c}, Timothy Mattson, Jos\'e Moreira}

\date{Generated on \today\ at \currenttime\ EDT}

\newcommand{\kron}{\mathbin{\text{\footnotesize \textcircled{\raisebox{-0.3pt}{\footnotesize $\otimes$}}}}}
\newcommand{\grbarray}[1]{\bm{#1}}
\renewcommand{\vector}[1]{{\bf #1}}
\renewcommand{\matrix}[1]{{\bf #1}}
\renewcommand{\arg}[1]{{\sf #1}}
\newcommand{\zip}{{\mbox{zip}}}
\newcommand{\zap}{{\mbox{zap}}}
\newcommand{\ewiseadd}{{\mbox{\bf ewiseadd}}}
\newcommand{\ewisemult}{{\mbox{\bf ewisemult}}}
\newcommand{\mxm}{{\mbox{\bf mxm}}}
\newcommand{\vxm}{{\mbox{\bf vxm}}}
\newcommand{\mxv}{{\mbox{\bf mxv}}}
\newcommand{\gpit}[1]{{\sf #1}}
\newcommand{\ie}{{i.e.}}
\newcommand{\eg}{{e.g.}}
\newcommand{\nan}{{\sf NaN}}
\newcommand{\nil}{{\bf nil}}
\newcommand{\ifif}{{\bf if}}
\newcommand{\ifthen}{{\bf then}}
\newcommand{\ifelse}{{\bf else}}
\newcommand{\ifendif}{{\bf endif}}
\newcommand{\zero}{{\bf 0}}
\newcommand{\one}{{\bf 1}}
\newcommand{\true}{{\sf true}}
\newcommand{\false}{{\sf false}}
\newcommand{\syntax}{{C++ Syntax}}

\newcommand{\code}[1]{\mintinline{cpp}{#1}}

\newcommand{\codet}[1]{\texttt{#1}}

\newcommand{\codetlink}[2]{{\color{cyan}\hyperlink{#1}{\texttt{#2}}}}

\newcommand{\idxname}{\texttt{grb::index\_t}}

\newcommand{\Dinn}{\mbox{$D_{in}$}}
\newcommand{\Din}[1]{\mbox{$D_{in_{#1}}$}}
\newcommand{\Dout}{\mbox{$D_{out}$}}

\newcommand{\bDinn}{\mbox{$\mathbf{D}_{in}$}}
\newcommand{\bDin}[1]{\mbox{$\mathbf{D}_{in_{#1}}$}}
\newcommand{\bDout}{\mbox{$\mathbf{D}_{out}$}}

\newcommand{\aydin}[1]{{{\color{orange}[Aydin: #1]}}}
\newcommand{\scott}[1]{{{\color{violet}[Scott: #1]}}}
\newcommand{\tim}[1]{{{\color{teal}[Tim: #1]}}}
\newcommand{\jose}[1]{{{\color{red}[Jose: #1]}}}
\newcommand{\ben}[1]{{{\color{blue}[Ben: #1]}}}

%\newcommand{\aydin}[1]{}
%\newcommand{\scott}[1]{}
%\newcommand{\tim}[1]{}
%\newcommand{\jose}[1]{}
%\newcommand{\ben}[1]{}

%\carl{testing}
%\scott{testing}
%\aydin{testing}
%\tim{testing}
%\jose{testing}
%\ajy{testing}

\renewcommand{\comment}[1]{{}}
\newcommand{\glossBegin}{\begin{itemize}}
\newcommand{\glossItem}[1]{\item \emph{#1}: }
\newcommand{\glossEnd}{\end{itemize}}

\setlength{\parskip}{0.5\baselineskip}
\setlength{\parindent}{0ex}

\maketitle


\renewcommand{\thefootnote}{\arabic{footnote}}
\setcounter{footnote}{0}

\vfill

Copyright \copyright\ 2020 Carnegie Mellon University, The Regents 
of the University of California, through Lawrence Berkeley National 
Laboratory (subject to receipt of any required approvals from the 
U.S. Dept. of Energy), the Regents of the University of California 
(U.C. Berkeley), Intel Corporation, International Business Machines 
Corporation, and Massachusetts Institute of Technology Lincoln
Laboratory. 

Any opinions, findings and conclusions or recommendations expressed in 
this material are those of the author(s) and do not necessarily reflect 
the views of the United States Department of Defense, the United States 
Department of Energy, Carnegie Mellon University, the Regents of the 
University of California, Intel Corporation, or the IBM Corporation.  

NO WARRANTY. THIS MATERIAL IS FURNISHED ON AN AS-IS BASIS. THE COPYRIGHT 
OWNERS AND/OR AUTHORS MAKE NO WARRANTIES OF ANY KIND, EITHER EXPRESSED 
OR IMPLIED, AS TO ANY MATTER INCLUDING, BUT NOT LIMITED TO, WARRANTY OF 
FITNESS FOR PURPOSE OR MERCHANTABILITY, EXCLUSIVITY, OR RESULTS OBTAINED 
FROM USE OF THE MATERIAL. THE COPYRIGHT OWNERS AND/OR AUTHORS DO NOT MAKE 
ANY WARRANTY OF ANY KIND WITH RESPECT TO FREEDOM FROM PATENT, TRADE MARK, 
OR COPYRIGHT INFRINGEMENT.

\vspace{1.5cm}

\vspace{2cm}
%{\Large This version is a definitive release of the GraphBLAS C API
%specification. As of the date of this document, at least two independent
%and functionally complete implementations are available.}

{\Large This version is a provisional release of the GraphBLAS C++ API specification.
Once two functionally complete reference implementations are available, we
will remove the "provisional" clause.}


\vspace{1.5cm}


%[Distribution Statement A] This material has been approved for public release and unlimited distribution.  
%Please see Copyright notice for non-US Government use and distribution.

Except as otherwise noted, this material is licensed under a Creative Commons Attribution 4.0 license (\href{http://creativecommons.org/licenses/by/4.0/legalcode}{http://creativecommons.org/licenses/by/4.0/legalcode}), 
and examples are licensed under the BSD License (\href{https://opensource.org/licenses/BSD-3-Clause}{https://opensource.org/licenses/BSD-3-Clause}).

%\begin{abstract}
%\end{abstract}

\vfill
\pagebreak

%-----------------------------------------------------------------------------
\tableofcontents
\vfill
\pagebreak

%-----------------------------------------------------------------------------
\addcontentsline{toc}{section}{List of Tables}
\listoftables
\vfill
\pagebreak

\addcontentsline{toc}{section}{List of Figures}
\listoffigures
\vfill
\pagebreak

%-----------------------------------------------------------------------------

\section*{Acknowledgments}
\addcontentsline{toc}{section}{Acknowledgments}

This document represents the work of the people who have served on the C++ API
Subcommittee of the GraphBLAS Forum.

Those who served as C++ API Subcommittee members for GraphBLAS 1.x.x are (in alphabetical order):
\begin{itemize}
\item Benjamin Brock (UC Berkeley)
\item Ayd\i n Bulu\c{c} (Lawrence Berkeley National Laboratory)
\item Timothy G. Mattson (Intel Corporation)
\item Scott McMillan (Software Engineering Institute at Carnegie Mellon University)
\item Jos\'e Moreira (IBM Corporation)
\end{itemize}

The GraphBLAS specification is based upon work funded and supported in part by:
\begin{itemize}
\item The Department of Energy Office of Advanced Scientific Computing Research under contract number DE-AC02-05CH11231
\item Intel Corporation
\item Department of Defense under Contract No. FA8702-15-D-0002 with Carnegie Mellon University for the operation of the Software Engineering Institute [DM-0003727, DM19-0929]
\item International Business Machines Corporation
\item Department of Defense under contract No. W911QX-12-C-0059, L-3 Data Tactics subcontract SCT-14-004 with University of California, Davis
\item NSF Graduate Research Fellowship under Grant No. DGE 1752814 and by the NSF under Award No. 1823034 with the University of California, Berkeley
\end{itemize}

The following people provided valuable input and feedback during the development of the specification (in alphabetical order):
Hollen Barmer, Tim Davis, Jeremy Kepner, Peter Kogge, Manoj Kumar, Andrew Mellinger, 
Maxim Naumov, Nancy M. Ott, Ping Tak Peter Tang, Michael Wolf, Carl Yang, Albert-Jan Yzelman.
\vfill
\pagebreak

%-----------------------------------------------------------------------------

\chapter{Introduction}

The GraphBLAS standard defines a set of matrix and vector operations 
based on semi-ring algebraic structures.  
These operations can be used
to express a wide range of graph algorithms.   This document 
defines the C++ binding to the GraphBLAS standard.   We refer to 
this as the {\it GraphBLAS C++ API} (Application Programming Interface).   

The GraphBLAS C++ API is built on a collection of   
objects exposed to the C++ programmer as opaque data types. 
Functions that manipulate these
objects are referred to as {\it methods}.  These methods fully define the 
interface to GraphBLAS objects to create or destroy them, modify their 
contents, and copy the contents of opaque objects into non-opaque objects; the 
contents of which are under direct control of the programmer.
\scott{We need to adopt C++ terminology, class methods or member functions, and free functions for operations}.

The GraphBLAS C++ API is designed to work with C++17/14/11 (ISO/IEC xxxx:xxx) 
\scott{need to figure out which version of the language we are going to require}
extended with {\it static type-based} and {\it number of parameters-based}.  
Furthermore, the standard assumes programs using the GraphBLAS C++ API
will execute on hardware that supports floating point arithmetic
such as that defined by the IEEE~754 (IEEE 754-2008) standard. 

The remainder of this document is organized as follows:
\begin{itemize}
\item Chapter~\ref{Chp:Concepts}: Basic Concepts
\item Chapter~\ref{Chp:NamedRequirements}: Named Requirements
\item Chapter~\ref{Chp:Classes}: Classes
\item Chapter~\ref{Chp:Functions}: Functions
\item Chapter~\ref{Chp:Operations}: Operations
\item Appendix~\ref{Chp:RevHistory}: Revision History
\item Appendix~\ref{Chp:Examples}: Examples
\end{itemize}

%=============================================================================
%=============================================================================

\chapter{Basic Concepts}
\label{Chp:Concepts}

% ========================================================================
\section{Glossary}

\subsection{GraphBLAS API basic definitions}

\subsection{GraphBLAS objects and their structure}

\subsection{Algebraic structures used in the GraphBLAS}

Operators must comply with the style dictated for callables from {\sf <functional>} 
header file and lambdas.  Recommend C++14 approach that has deprecated use of {\sf result\_type}
typedefs and instead use
\begin{verbatim}
   decltype(binaryop(std::declval<typename LHS::ScalarType>(),
                     std::declval<typename RHS::ScalarType>()))
\end{verbatim}


GBTL demostrates how lambas and things like std::bind objects can be passed 
as arguments in place of UnaryOp and BinaryOp (they follow the Callable 
concept).  

Monoids and Semirings do not follow the Callable Concept.  These are
structs with methods not restricted to operator()().


\subsection{The execution of an application using the GraphBLAS C API}

\subsection{GraphBLAS methods: behaviors and error conditions}


% ========================================================================
\section{Notation}


% ========================================================================
\section{Algebraic and Arithmetic Foundations}


% ========================================================================
\section{GraphBLAS Opaque Objects}

Not GrB\_Type? (should not exist)\\
Not GrB\_Descriptor (does not exist)

While we have the concepts of UnaryOp, BinaryOp, Monoid, and Semiring we 
do not need to have such classes.  GBTL demostrates how lambas and things 
like std::bind objects can be passed as arguments in place of UnaryOp and
BinaryOp (they follow the Callable concept).  Monoids and Semirings do not 
follow the Callable Concept.

Right now only Vector and Matrix for sure.

There is a frontend Matrix and Vector class that take the place of C handles.

% ========================================================================
\section{Domains}


% ========================================================================
\section{Operators and Associated Functions}

While we have the concepts of UnaryOp, BinaryOp, Monoid, and Semiring we do not have such classes

See GBTL's {\sf algrebra.hpp} for demonstration of all "predefined" operators.


% ========================================================================
\section{Indices, Index Arrays, and Scalar Arrays}

Unlike the C API, GraphBLAS Vectors and Matrices in C++ can be templatized on
the index type and so can take on any unsigned integer type; therefore, there is
no equivalent to GrB\_Index.

Need to deal with mismatched index types?

% ========================================================================
\section{Execution Model}

\subsection{Execution modes}

\subsection{Thread safety}
\label{Sec:ThreadSafety}


% ========================================================================
\section{Error Model}
\label{Sec:ErrorModel}

Depending on support for execution models we may or may not have two 
categories of errors: API and execution errors.

All runtime errors are reported by throwing exceptions.  All exceptions generated
by the library inherit from {\sf grb::exception} that (in turn) inherits 
from {\sf std::exception}.

Exceptions subclassed from {\sf grb::exception}:

\begin{itemize}
\item {\sf logic\_error} base class for API errors.
    \begin{itemize}
    \item {\sf dimension\_mismatch}
    \end{itemize}
\item {\sf runtime\_error} base class for execution errors.
    \begin{itemize}
    \item {\sf panic\_error}
    \item {\sf bad\_alloc}... should we use {\sf std::bad\_alloc} instead?
    \item {\sf invalid\_object}
    \end{itemize}
\end{itemize}

% ========================================================================
\section{Named Requirements}
This section contains \textit{named requirements} describing the requirements for
various types to be used by the C++ GraphBLAS standard.
GraphBLAS implementations may choose to formalize these named requirements with
C++ concepts.  Regardless, C++ GraphBLAS implementers are responsible for
ensuring that their implementations comply with these named requirements, and
users who pass in custom objects to the GraphBLAS API are responsible for
ensuring that their types satisfy the corresponding named requirements.

\subsection{Binary Operators}
An object whose type fulfills the \textit{BinaryOperator} requirement is a C++
\textit{Callable} that can be called with two arguments, returning a single
value.  We call the type of the lefthand argument passed in to the binary operator
\texttt{T} and the type of the righthand argument \texttt{U}.  A callable is a
valid binary operator for types \texttt{T} and \texttt{U} if it is callable with
arguments of those types, returning an object of type \texttt{V}.

\begin{tabularx}{\textwidth}{l l X}
\textbf{Expression} & \textbf{Return Type} & \textbf{Requirements}\\
\hline
\texttt{binary\_op(T(), U())} & \texttt{V} & \texttt{binary\_op} is callable \texttt{T} $\times$ \texttt{U} $\rightarrow$ \texttt{V}.\\
\end{tabularx}

\subsection{Monoids}
An object whose type fulfills the \textit{Monoid} requirement is a callable that
fulfills the \textit{BinaryOperator} requirement in addition to two other requirements.
We say that an object is a monoid on type \texttt{T} if:

\begin{enumerate}
   \item The monoid is a valid \textit{BinaryOperator} on type \texttt{T} and has a return value of type \texttt{T}, \texttt{T} $\times$ \texttt{T} $\rightarrow$ \texttt{T}.
   \item The monoid has a method \texttt{identity} that is callable with template parameter \texttt{T}, returning an object of type \texttt{T} which is the identity for the binary operator on that type.
\end{enumerate}

\begin{tabularx}{\textwidth}{l l X}
\textbf{Expression} & \textbf{Return Type} & \textbf{Requirements}\\
\hline
\texttt{monoid(T(), T())} & \texttt{T} & \texttt{monoid} is callable \texttt{T} $\times$ \texttt{T} $\rightarrow$ \texttt{T}.\\
\hline
\texttt{monoid.identity<T>()} & \texttt{T} & \texttt{identity()} method callable with template parameter \texttt{T}.\\
\end{tabularx}

% ========================================================================
\chapter{Named Requirements}
\label{Chp:NamedRequirements}

This section contains \textit{named requirements} describing the requirements for
various types to be used by the C++ GraphBLAS standard.
GraphBLAS implementations may choose to formalize these named requirements with
C++ concepts.  Regardless, C++ GraphBLAS implementers are responsible for
ensuring that their implementations comply with these named requirements, and
users who pass in custom objects to the GraphBLAS API are responsible for
ensuring that their types satisfy the corresponding named requirements.

% ========================================================================
\section{Unary Operators}

% ========================================================================
\section{Index Unary Operators}

% ========================================================================
\section{Binary Operators}
An object whose type fulfills the \textit{BinaryOperator} requirement is a C++
\textit{Callable} that can be called with two arguments, returning a single
value.  We call the type of the lefthand argument passed in to the binary operator
\texttt{T} and the type of the righthand argument \texttt{U}.  A callable is a
valid binary operator for types \texttt{T} and \texttt{U} if it is callable with
arguments of those types, returning an object of type \texttt{V}.

\begin{tabularx}{\textwidth}{l l X}
\textbf{Expression} & \textbf{Return Type} & \textbf{Requirements}\\
\hline
\texttt{binary\_op(T(), U())} & \texttt{V} & \texttt{binary\_op} is callable \texttt{T} $\times$ \texttt{U} $\rightarrow$ \texttt{V}.\\
\end{tabularx}

% ========================================================================
\section{Monoids}
An object whose type fulfills the \textit{Monoid} requirement is a callable that
fulfills the \textit{BinaryOperator} requirement in addition to two other requirements.
We say that an object is a monoid on type \texttt{T} if:

\begin{enumerate}
   \item The monoid is a valid \textit{BinaryOperator} on type \texttt{T} and has a return value of type \texttt{T}, \texttt{T} $\times$ \texttt{T} $\rightarrow$ \texttt{T}.
   \item The monoid has a method \texttt{identity} that is callable with template parameter \texttt{T}, returning an object of type \texttt{T} which is the identity for the binary operator on that type.
\end{enumerate}

\begin{tabularx}{\textwidth}{l l X}
\textbf{Expression} & \textbf{Return Type} & \textbf{Requirements}\\
\hline
\texttt{monoid(T(), T())} & \texttt{T} & \texttt{monoid} is callable \texttt{T} $\times$ \texttt{T} $\rightarrow$ \texttt{T}.\\
\hline
\texttt{monoid.identity<T>()} & \texttt{T} & \texttt{identity()} method callable with template parameter \texttt{T}.\\
\end{tabularx}

% ========================================================================
\section{GraphBLAS Container? Matrix? Vector? Scalar?}

Not sure we need to define this.  Not sure we should have separate Matrix, Vector, 
and Scalar containers. I think these are the opaque data structures that implementers create.

\scott{Does something belong here or is it all in the containers section?}

\paragraph{Requirements}
\begin{itemize} \itemsep0em
\item \texttt{C} container type
\item \texttt{ScalarT} element type
\item \texttt{IndexT} indexing type (location of elements)
\end{itemize}

\paragraph{Types}

\begin{tabularx}{\textwidth}{l l X}
\textbf{Name} & \textbf{Type} & \textbf{Notes}\\
\hline
\texttt{value\_type} & \texttt{ScalarT}  & Erasable?.\\ \hline
\texttt{reference}  & \texttt{ScalarT\&} & Do we need this? \\ \hline
\texttt{const\_reference}  & \texttt{const ScalarT\&} & Do we need this? \\ \hline
\texttt{iterator}  & & \\ \hline
\texttt{const\_iterator}  & & \\ \hline
\texttt{different\_type}  & & \\ \hline
\texttt{size\_type}  & &
\end{tabularx}

% ========================================================================
\section{VectorView}

An object whose type fulfills the \textit{VectorView} requirement is (loosely) a 
reference to an opaque \textit{GraphBLAS Container} that has one dimension and a 
number of stored elements.  A \textit{GraphBLAS Container} is an object used to store
other objects, \textit{GraphBLAS Scalars}, that takes care of the management of the
memory used by these objects. \scott{Wording from C++ Container named requirements.}

\scott{Should the VectorView only define enough interface to perform any runtime API error checks?}

\scott{Should VectorView provide a reference to the internal container?  This is an implementation specific mechanism that does not need to be specified.}

\paragraph{Types}

\begin{tabularx}{\textwidth}{l l X}
\textbf{Name} & \textbf{Type} & \textbf{Notes}\\
\hline
\texttt{value\_type} & \texttt{ScalarT}  & The type of objects stored in the container.\\ \hline
\texttt{index\_type} & \texttt{IndexT}   & Used to reference locations in the container \\ \hline
\texttt{size\_type}  & ??                & Can hold the number of elements stored in the container.
\end{tabularx}

\paragraph{Methods and operators}

\begin{tabularx}{\textwidth}{l l X}
\textbf{Expression} & \textbf{Return Type} & \textbf{Requirements}\\
\hline
\texttt{size()} & \texttt{index\_type}  & The size of the dimension of the vector.\\ \hline
\texttt{nvals()} & \texttt{size\_type}  & The number of elements stored in the vector. \scott{this is not required}\\ \hline
constructors? & & \\
\end{tabularx}

A \emph{Vector} object is a valid \emph{VectorView} object

% ========================================================================
\section{ConstVectorView}

Is the same as a \textit{VectorView} except that the underlying opaque container within cannot be
mutated by calls to GraphBLAS operations.  A \emph{Vector}, \emph{ConstVector} or 
\emph{VectorView} object is a valid \emph{ConstVectorView} objects.

% ========================================================================
\section{MatrixView}
An object whose type fulfills the \textit{MatrixView} requirement is (loosely) a 
reference to an opaque \textit{GraphBLAS Container} that has two dimensions and a 
number of stored elements.  A \textit{GraphBLAS Container} is an object used to store
other objects, \textit{GraphBLAS Scalars}, that takes care of the management of the
memory used by these objects. \scott{Wording from C++ Container named requirements.}

\scott{Should the MatrixView only define enough interface to perform any runtime API error checks?}

\scott{Should MatrixView provide a reference to the internal container?}

\paragraph{Types}

\begin{tabularx}{\textwidth}{l l X}
\textbf{Name} & \textbf{Type} & \textbf{Notes}\\
\hline
\texttt{value\_type} & \texttt{ScalarT}  & The type of objects stored in the container.\\ \hline
\texttt{index\_type} & \texttt{IndexT}   & Used to reference locations in the container \\ \hline
\texttt{size\_type}  & ??                & Can hold the number of elements stored in the container.
\end{tabularx}

\paragraph{Methods and operators}

\begin{tabularx}{\textwidth}{l l X}
\textbf{Expression} & \textbf{Return Type} & \textbf{Requirements}\\
\hline
\texttt{nrows()} & \texttt{index\_type} & The size of the first dimension of the matrix.\\ \hline
\texttt{ncols()} & \texttt{index\_type} & The size of the second dimension of the matrix.\\ \hline
\texttt{nvals()} & \texttt{size\_type}  & The number of elements stored in the matrix. \scott{this is not required}\\ \hline
constructors? & & \\
\end{tabularx}

A \emph{Matrix} object is a valid \emph{MatrixView} object

% ========================================================================
\section{ConstMatrixView}

Is the same as a \textit{MatrixView} except that the underlying opaque container within cannot be
mutated by calls to GraphBLAS operations.  A \emph{Matrix}, \emph{ConstMatrix} or 
\emph{MatrixView} object is a valid \emph{ConstMatrixView} objects.

% ========================================================================
\section{Hints}

Talk about Hint and combiner here or somewhere else?

\chapter{Classes}
\label{Chp:Classes}


This section defines the classes that correspond the GraphBLAS objects.

% ========================================================================
\section{Operators}

\scott{All callable objects must be supported, including <functional> and lambdas.}

\scott{There are no separate "Types".  UnaryOp and BinaryOps are all callable objects.  Monoids and Semirings are class templates}

Are the named requirements enough?

Give a few examples of functions, functors and lambdas  that can be used where a unaryop, binaryop, monoid and semiring are needed.

\subsection{Predefined operators}

\scott{Is there a set of the predefined operators? If so, list them.}

\subsection{\sf grb::NoAccumulate}

% ========================================================================
\section{Monoids}

Do we define the concept of a Monoid here.  We don't "new" them, rather they
are passed as templated arguments to the backend implementation.

\scott{Likely only structs, with operator() + identity()? Do we support result\_type?}

\subsection{Predefined monoids}


% ========================================================================
\section{Semirings}

Do we define the concept of a Semiring here.  We don't "new" them, rather they
are passed as templated arguments to the backend implementation.


\scott{likely only structs with add90, mult(), and zero().  Do we support 
first/second\_argument\_type and result\_type?}

\subsection{Predefined semirings}

% ========================================================================
\section{Sequences}
\label{Sec:Sequences}

There should be other sequences that replace "arrays of indices" that can also be substituted.

This is also important for distributed implementations where enumerating all indices in a range can be too large.

\subsection{Predefined sequences}

\subsection{\sf grb::AllIndices}

{\sf grb::AllIndices} a.k.a {\sf GrB\_ALL} is an example of a compact sequence.

% ========================================================================
\section{Vectors: {\sf grb::vector} class}
\label{Sec:Vectors}

This section defines the minimum required public interface to the "API"  Vector 
class.  It satisfies \textit{VectorCollection}?

\scott{Implementers free to add to the public API so long as the named requirement is still satisfied.}

Definition:

\begin{verbatim}
template <typename T,
          typename I = std::size_t,
          typename Hint = grb::bitmap_sparse,
          typename Allocator = std::allocator<T>>
class vector;
\end{verbatim}

\paragraph{Member Types}

\begin{tabularx}{\textwidth}{l X}
\textbf{Member Type} & \textbf{Definition}\\
\hline
\codet{value\_type} or \codet{ScalarType} & \codet{T}.  The type of elements stored in the Vector. \textit{CopyAssignable}. \textit{CopyConstructible} (pre C++11?) \scott{Here is the statement from the C++ Standard Graph Library proposal: "The graph value type defined by the user.  It can be most valid C++ value type including class, struct, tuple, union, enum, array, reference or scalar value. If no value is needed then the empty\_value struct can be used."}\\
\hline
\texttt{index\_type} or \texttt{IndexType} & \texttt{I}, \scott{Should we restrict this to unsigned integer types?} \\
\hline
\texttt{size\_type} & Unsigned integer type (usually \texttt{std::size\_t}).  \scott{For vector, size\_type and index\_type could be same.} \\
\hline
\texttt{difference\_type} & Signed integer type (usually \texttt{std::ptrdiff\_t})\\
\hline
\texttt{hint\_type} & \texttt{Hint}\\
\hline
\texttt{allocator\_type} & \texttt{Allocator}\\
\hline
\texttt{reference} & \textit{VectorReference} to \texttt{value\_type}\\
\hline
\texttt{const\_reference} & \textit{VectorReference} to \texttt{const value\_type}\\
\hline
\texttt{pointer} & \textit{VectorIterator} to \texttt{value\_type}\\
\hline
\texttt{const\_pointer} & \textit{VectorIterator} to \texttt{const value\_type}\\
\hline
\texttt{iterator} & \textit{VectorIterator} to \texttt{value\_type}\\
\hline
\texttt{const\_iterator} & \textit{Vectoriterator} to \texttt{const value\_type}\\
\end{tabularx}

\paragraph{Member functions}

\begin{tabularx}{\textwidth}{l X}
\textbf{Construct, assign, and move}\\
\hline
\texttt{vector()} & Construct empty vector of size $0$, \scott{Should we " = delete" this constructor?} \\ 
\hline
\texttt{vector(index\_type size)} & Construct vector of size $\texttt{size}$. \scott{should zero be disallowed...throw?} \\
& \texttt{panic\_exception} with unknown internal error \\
& \texttt{bad\_alloc} if not enough memory \\
& \texttt{invalid\_value} is size is zero. \\
\hline
\texttt{vector(std::string fname)} & Construct vector object from an \scott{XXX (what format?)} file at filepath \texttt{fname}.\\
\hline
\texttt{vector(const vector\& other)} & Copy construct vector from \texttt{other}.\\
\hline
\texttt{vector(vector\&\& other)} & Move construct vector from \texttt{other}.\\
\hline
\texttt{vector\& operator=(const vector\& other)} & Assignment operator from \texttt{other}. \scott{Check signature} \\
\hline
\texttt{vector\& operator=(vector\&\& other)} & Move assignment operator from \texttt{other} return state of \texttt{other} is implementation defined?. \scott{Check signature. is this noexcept?} \\
\hline
\texttt{\~{}vector()} & Destructor. Releases all resource held by the container within this object including all implementation data structures related to this object. \scott{Not virtual} \\
\end{tabularx}

\begin{tabularx}{\textwidth}{l X}
\textbf{Iterators}\\
\hline
\texttt{iterator begin()} & Non-const iterator to the beginning\\
\hline
\texttt{const\_iterator begin() const} & Const iterator to the beginning\\
\hline
\texttt{iterator end()} & Non-const iterator to the end\\
\hline
\texttt{const\_iterator end() const} & Const iterator to the end\\
\end{tabularx}

\begin{tabularx}{\textwidth}{l X}
\textbf{Capacity and Dimensions }\\
\hline
\texttt{size\_type reserve(size\_type capacity)} & \scott{(reserve not supported by opacity?)}\\
\hline
\texttt{size\_type capacity() const} & \scott{(capacity not supported by opacity?)}\\
\hline
\texttt{size\_type nvals() const} & Number of values stored in the vector (size()?)\\
\hline
\texttt{index\_type size() const} & Size of vector (dimension()?)\\
\hline
\texttt{index\_type resize(index\_type) new\_size)} & Size of vector (dimension()?)\\
\end{tabularx}

\begin{tabularx}{\textwidth}{l X}
\textbf{Lookup}\\
\hline
\texttt{reference at(index\_type index)} & Access specific element with bounds checking\\
\hline
\texttt{const\_reference at(index\_type index) const} & Access specific element with bounds checking\\
\hline
\texttt{reference operator[](index\_type index)} & Access or insert a specified element\\
\hline
\texttt{const\_reference operator[](index\_type index) const} & Access a specified element\\
\hline
\texttt{iterator find(index\_type index)} & Find element at a specific index\\
\hline
\texttt{const\_iterator find(index\_type index) const} & Find element at a specific index\\
\end{tabularx}

\begin{tabularx}{\textwidth}{l X}
\textbf{Other}\\
\hline
\texttt{iterator remove(index\_type index)} & Annihilate element at a specific index \scott{erase(ix)?} \\
\hline
\texttt{void clear()} &  \\
\hline
\texttt{void build(RAIndexIterator  i\_it,}  & container version?? \\
\texttt{~~~~~~~~~~~RAValueIterator  v\_it,}  & "not empty" exception?? index out of bounds exception?? \\
\texttt{~~~~~~~~~~~size\_type       nval,}  & ?? \\
\texttt{~~~~~~~~~~~BinaryOperator \&dup)}   & reference?? \\
\hline
\texttt{void build(vector<IndexType>  const \&indices,}  & const or move? bad sizes exception\\
\texttt{~~~~~~~~~~~vector<ScalarType> const \&values,}  & "not empty" exception?? index out of bounds exception?? \\
\texttt{~~~~~~~~~~~BinaryOperator \&dup)}   & reference?? \\
\hline
\texttt{void extractTuples()} & What is the C++ equivalent? \\
\texttt{void wait()} & What is the C++ equivalent? \\
\texttt{void error()} & What is the C++ equivalent? \\
\end{tabularx}

How should exceptions be documented?

\scott{Do we support void Scalar type for matrices and vectors?}

% ========================================================================
\section{Matrices: {\sf grb::matrix} class}
\label{Sec:Matrices}

\scott{define index\_t, shape\_t, location\_t, or whatever.}

Definition:

\begin{verbatim}
template <typename T,
          typename I = std::size_t,
          typename Hint = grb::sparse,
          typename Allocator = std::allocator<T>>
class matrix;
\end{verbatim}

\paragraph{Member Types}

\begin{tabularx}{\textwidth}{l X}
\textbf{Member Type} & \textbf{Definition}\\
\hline
\texttt{value\_type} & \texttt{T} \\
\hline
\texttt{index\_type} & \texttt{I} \\
\hline
\texttt{size\_type} & Unsigned integer type (usually \texttt{std::size\_t})\\
\hline
\texttt{difference\_type} & Signed integer type (usually \texttt{std::ptrdiff\_t})\\
\hline
\texttt{hint\_type} & \texttt{Hint}\\
\hline
\texttt{allocator\_type} & \texttt{Allocator}\\
\hline
\texttt{reference} & \textit{MatrixReference} to \texttt{value\_type}\\
\hline
\texttt{const\_reference} & \textit{MatrixReference} to \texttt{const value\_type}\\
\hline
\texttt{pointer} & \textit{MatrixIterator} to \texttt{value\_type}\\
\hline
\texttt{const\_pointer} & \textit{MatrixIterator} to \texttt{const value\_type}\\
\hline
\texttt{iterator} & \textit{MatrixIterator} to \texttt{value\_type}\\
\hline
\texttt{const\_iterator} & \textit{Matrixiterator} to \texttt{const value\_type}\\
\end{tabularx}

\paragraph{Member Functions}

\begin{tabularx}{\textwidth}{l X}
\hline
\codetlink{constructors}{(constructor)} & Construct the matrix object\\
\codetlink{constructors}{(destructor)} & Destruct the matrix object\\
\codetlink{assign_ops}{operator=} & Assign values to the matrix object\\
\end{tabularx}

\begin{tabularx}{\textwidth}{l X}
\textbf{Iterators}\\
\hline
\texttt{iterator begin()} & Non-const iterator to the beginning\\
\hline
\texttt{const\_iterator begin() const} & Const iterator to the beginning\\
\hline
\texttt{iterator end()} & Non-const iterator to the end\\
\hline
\texttt{const\_iterator end() const} & Const iterator to the end\\
\end{tabularx}

\begin{tabularx}{\textwidth}{l X}
\textbf{Capacity and Dimensions}\\
\hline
\texttt{size\_type size() const} & Number of stored values\\
\hline
\texttt{index\_t shape() const} & Dimensions of matrix \scott{shape\_t}\\
\end{tabularx}

\begin{tabularx}{\textwidth}{l X}
\textbf{Lookup}\\
\texttt{reference at(index\_t index)} & Access specific element with bounds checking\\
\hline
\texttt{const\_reference at(index\_t index) const} & Access specific element with bounds checking\\
\hline
\texttt{reference operator[](index\_t index)} & Access or insert a specified element\\
\hline
\texttt{const\_reference operator[](index\_t index) const} & Access a specified element\\
\hline
\texttt{iterator find(index\_t index)} & Find element at a specific index\\
\hline
\texttt{const\_iterator find(index\_t index) const} & Find element at a specific index\\
\end{tabularx}

\scott{Do we support void Scalar type for matrices and vectors?}
\ben{I don't think we should.  It doesn't work with \codet{std::vector}, etc.  I would say define a \codet{grb::no\_value} or something.}

\subsection{Constructors and Destructors}
\hypertarget{constructors}{This section} defines constructors and destructors for \codet{grb::matrix}.

\begin{minted}{c++}
  matrix();                                      (1)

  matrix(grb::index_t dim);                      (2)

  matrix(std::string fname);                     (3)

  matrix(const matrix& other);                   (4)

  matrix(matrix&& other);                        (5)

  ~matrix();                                     (6)
\end{minted}

%\texttt{matrix()} & Construct empty matrix of dimension $0 \times 0$\\
%\hline
%\texttt{matrix(index\_t dim)} & Construct matrix of dimension $\texttt{dim[0]} \times \texttt{dim[1]}$\\
%\hline
%\texttt{matrix(\scott{shape\_t} dim)} & Construct matrix of dimension $\texttt{dim[0]} \times \texttt{dim[1]}$\\
%\hline
%\texttt{matrix(std::initializer\_list<{\color{red} index\_type}> dim)} & Construct matrix of dimension $\texttt{*dim.begin()} \times \texttt{*(++dim.begin)}$\\
%\hline
%\texttt{matrix(std::string fname)} & Construct matrix object from Matrix Market file \scott{(don't limit to MTX.)} at filepath \texttt{fname}.\\
%\hline
%\texttt{matrix(const matrix\& other)} & Copy construct matrix from \texttt{other}.\\
%\hline
%\texttt{matrix(matrix\&\& other)} & Move construct matrix from \texttt{other}.\\
%\texttt{\~{}matrix()} & Destructor. Releases all resource held by the container within this object including all implementation data structures related to this object. \scott{Not virtual} \\

\subsection{Assignment Operators}
\hypertarget{assign_ops}{This section} defines assignment operators for \codet{grb::matrix}.

\begin{minted}{c++}
  matrix& operator=(const matrix& other);        (1)

  matrix& operator=(matrix&& other);             (2)
\end{minted}

\ben{Should we add assign to a matrix of a different type? (I think yes, with default casting.)  What about a view? (I also say yes, but you may not want to support it.)}

\ben{Should we add assign to a single element? (I think no, it's too ambiguous.)}

%\hline
%\texttt{matrix\& operator=(const matrix\& other)} & Copy assignment operator \texttt{other}.\\
%\hline
%\texttt{matrix\& operator=(matrix\&\& other)} & Move assignment operator \texttt{other}.\\
%\hline


% ========================================================================
\section{Masks}
\label{Sec:Masks}



\subsection{\sf grb::NoMask}

There is a predefined object that satisfies \textit{VectorMask} and \textit{MatrixMask}
requirements that basically removes the use of a mask from the operation. It
is mathematically equivalent to sending in a "full" mask.

\chapter{Functions}
\label{Chp:Functions}

A chapter for free functions that are not the GraphBLAS primitive operations.

%-----------------------------------------------------------------------------
%-----------------------------------------------------------------------------
\section{Views}

The C++ API specification supports a number of ``view'' types that wrap various
GraphBLAS containers (collections) to flag modifications of the collections
during calls to GraphBLAS functions.

%-----------------------------------------------------------------------------
\subsection{{\sf transpose}: Transpose view (input matrices only)}

Only for matrices.  
\scott{We do not need to specify the TransposeView class.  It is an
unspecified implementation detail.}

\paragraph{\syntax}

\begin{minted}{c++}
        template <typename MatrixType>
        /*unspecified*/ transpose(MatrixType const &mat);
\end{minted}

\scott{The "unspecified" style is taken from std::bind documentation: \\
https://en.cppreference.com/w/cpp/utility/functional/bind.  This allows it to be implementation dependent without specifying an auto return type.}

\paragraph{Parameters}

\begin{itemize}%[leftmargin=1.1in]
    \item[{\sf mat}] ({\sf IN}) A \emph{ConstMatrix}. A GraphBLAS Matrix object.  Will not be modified.
\end{itemize}

\paragraph{Return Value}

This method returns an object satisfying \emph{ConstMatrixView}.  The specific type is
implementation-dependent.

\paragraph{Description}

This function is intended to be used at the call site of GraphBLAS operations that take, 
as input, a GraphBLAS matrix that could be optionally transposed.  The function returns an
object that satisfies the \emph{ConstMatrixView} concept meaning it can only be used as an
{\sf IN} parameter that will not be modified.

\paragraph{Example}

\begin{minted}{c++}
        grb::Matrix A<float>({5, 10});
        grb::Matrix B<float>({5, 10});
        grb::Matrix C<float>({5, 5});
        // ...
        grb::mxm(C, grb::NoMask(), grb::NoAccumulate(), 
                 grb::PlusTimesSemiring<float>(), A, transpose(B));
\end{minted}


%-----------------------------------------------------------------------------
\subsection{{\sf structure}: Structure view (Masks (ConstContainers) only)}

Only for masks.  
\scott{We do not need to specify the StructureView class.  It is an
unspecified implementation detail.}

\paragraph{\syntax}

\begin{minted}{c++}
        template <typename MaskType>
        /*unspecified*/ structure(MaskType const &mask);
\end{minted}

\begin{itemize}%[leftmargin=1.1in]
    \item[{\sf mask}] ({\sf IN}) A GraphBLAS \emph{Mask} object. It is 
    overloaded to accept either a \emph{ConstMatrix} or \emph{ConstVector} 
    object.  The underlying object will not be modified.
\end{itemize}

\paragraph{Return Value}

This method returns an object satisfying a \emph{MaskView} (\emph{ConstMatrixView} or 
\emph{ConstVectorView}). If {\sf mask} is a \emph{ConstMatrix} type as input, 
it returns a \emph{ConstMatrixView}.  If {\sf mask} is a \emph{ConstVector} 
type as input, it returns a \emph{ConstVectorView}.  \scott{Do we need a specific
named requirement called MaskView?}

\paragraph{Description}

This function is intended to be used with masks at the call site of GraphBLAS 
operations that take, as input, a GraphBLAS matrix or vector mask.  The function 
returns an object that can be used by GraphBLAS operations that take a mask 
parameter as an {\sf IN} parameter (that will not be modified) and support the 
structure-only interpretation.

\paragraph{Example}

\begin{minted}{c++}
        grb::Matrix A<float>({5, 10});
        grb::Matrix B<float>({10, 5});
        grb::Matrix C<float>({5, 5});
        grb::Matrix M<bool>({5, 5});
        grb::Vector u<float>({10});
        grb::Vector w<float>({5});
        grb::Vector m<bool>({5});
        
        // ...
        grb::mxm(C, structure(M), grb::NoAccumulate(), 
                 grb::PlusTimesSemiring<float>(), A, B);
        // ...
        grb::mxv(w, structure(m), grb::NoAccumulate(), 
                 grb::PlusTimesSemiring<float>(), A, u);
\end{minted}


%-----------------------------------------------------------------------------
\subsection{{\sf complement}: Complement view (Masks (ConstContainers) and MaskViews only)}

Only for masks.  
\scott{We do not need to specify the ComplementView class.  It is an
unspecified implementation detail.}

\paragraph{\syntax}

\begin{minted}{c++}
        template <typename MaskType>
        /*unspecified*/ complement(MaskType const &mask);
\end{minted}

\begin{itemize}%[leftmargin=1.1in]
    \item[{\sf mask}] ({\sf IN}) A GraphBLAS \emph{Mask} or a \emph{MaskView}
    object (specifically what is output by structure() method). It is 
    overloaded to accept either a \emph{ConstMatrix} or \emph{ConstVector} 
    objects or their respective structure views.  The underlying object will 
    not be modified.
\end{itemize}

\paragraph{Return Value}

This method returns an object satisfying a \emph{MaskView} (\emph{ConstMatrixView} or 
\emph{ConstVectorView}). If {\sf mask} is a \emph{ConstMatrix} or \emph{ConstMatrixView} type as input, 
it returns a \emph{ConstMatrixView}.  If {\sf mask} is a \emph{ConstVector} or  \emph{ConstVectorView}
type as input, it returns a \emph{ConstVectorView}.  \scott{Do we need a specific
named requirement called MaskView?}

\paragraph{Description}

This function is intended to be used with masks at the call site of GraphBLAS 
operations that take, as input, a GraphBLAS matrix or vector mask.  The function 
returns an object that can be used by GraphBLAS operations that take a mask 
parameter as an {\sf IN} parameter (that will not be modified) and support the 
interpretation of the complement of the object that is input to this method.

Unlike the {\sf structure()} function that can only accept \emph{ConstMatrix} or 
\emph{ConstVector}, this {\sf complement()} function must also be able to take as
input the object returned by {\sf structure()} to allow operations to support the
use of structural-complement masks (i.e., complements of the structure of a mask).

\paragraph{Examples}

\begin{minted}{c++}
        grb::Matrix A<float>({5, 10});
        grb::Matrix B<float>({10, 5});
        grb::Matrix C<float>({5, 5});
        grb::Matrix M<bool>({5, 5});
        grb::Vector u<float>({10});
        grb::Vector w<float>({5});
        grb::Vector m<bool>({5});
        // ...
        grb::mxm(C, complement(M), grb::NoAccumulate(), 
                 grb::PlusTimesSemiring<float>(), A, B);
        // ...
        grb::mxm(C, complement(structure(M)), grb::NoAccumulate(), 
                 grb::PlusTimesSemiring<float>(), A, B);
        // ...
        grb::mxv(w, complement(m), grb::NoAccumulate(), 
                 grb::PlusTimesSemiring<float>(), A, u);
        // ...
        grb::mxv(w, complement(structure(m)), grb::NoAccumulate(), 
                 grb::PlusTimesSemiring<float>(), A, u);

        // Does not compile:
        grb::mxv(w, structure(complement(m)), grb::NoAccumulate(), // ERROR
                 grb::PlusTimesSemiring<float>(), A, u);
\end{minted}

% \mintinline{cpp}{code}
%=============================================================================
\chapter{GraphBLAS Operations}
\label{Ch:Operations}

%-----------------------------------------------------------------------------
\section{{\sf grb::mxm}: matrix-matrix multiply}

\paragraph{\syntax}

\begin{minted}{c++}
    template<typename CMatrixType,
             typename MaskType,
             typename AccumulatorType,
             typename SemiringType,
             typename AMatrixType,
             typename BMatrixType>
    void mxm(CMatrixType            &C,
             MaskType         const &Mask,
             AccumulatorType         accum,   // pass by value or const&?
             SemiringType            op,      // pass by value or const&?
             AMatrixType      const &A,
             BMatrixType      const &B,
             OutputControlEnum       outp = MERGE);  //or bool replace_flag = false);

    // ...or...
    template<typename CMatrixType,
             typename MaskType,
             typename AccumulatorType,
             typename ReduceType,
             typename MapType
             typename AMatrixType,
             typename BMatrixType>
    void mxm(CMatrixType            &C,
             MaskType         const &Mask,
             AccumulatorType         accum,   // pass by value or const&?
             ReduceType              reduce,  // pass by value or const&?
             MapType                 map,     // pass by value or const&?
             AMatrixType      const &A,
             BMatrixType      const &B,
             OutputControlEnum       outp = MERGE);  //or bool replace_flag = false);
\end{minted}

Multiplies two GraphBLAS matrices using the operators and identity defined by a GraphBLAS semiring. An optional accumulator and write mask can also be specified. The result is stored in third GraphBLAS matrix.

\begin{enumerate}
\item Any notes go here.
\end{enumerate}

\paragraph{Parameters}

\begin{itemize}[leftmargin=1.1in]
    \item[{\sf C}]    ({\sf INOUT}) A GraphBLAS \emph{Matrix} type. On input,
    the matrix provides values that may be accumulated with the result of the
    matrix product.  On output, the matrix holds the results of the
    operation.

    \item[{\sf Mask}] ({\sf IN}) An optional ``write'' mask (a \emph{ConstMatrixView}) that controls which
    results from this operation are stored into the output matrix {\sf C}. The 
    mask dimensions must match those of the matrix {\sf C}. 
    \scott{
    To complement/invert the logic of a mask, wrap the mask in a complement view by calling \code{grb::complement(Mask)}.
    To use the structure of this matrix only, wrap the mask in a structure view by calling \code{grb::structure(Mask)}.
    These views can be compose to get the complement of the structure of a mask by nesting these calls:
    \code{grb::complement(grb::structure(Mask))} (Note that \code{grb::structure(grb::complement(Mask))} is invalid and should not compile).
    If it is not wrapped in the \code{grb::structure} view, the domain 
    of the {\sf Mask} matrix must be of type that can be compared to \code{bool}.}
    If the default
    mask is desired (\ie, logically, a mask that is all {\sf true} with the dimensions of {\sf C}), 
    {\sf grb::no\_mask} should be passed in for this argument.
    \scott{What should be passed?  \code{grb::no_mask} or \code{grb::NoMask()}}

    \item[{\sf accum}] ({\sf IN}) An optional binary operator used for accumulating
    entries into existing {\sf C} entries.  If assignment rather than accumulation is
    desired, \code{grb::no_accum} should be specified. \scott{What should be passed?
    \code{grb::no_accum} or \code{grb::NoAccumulate()}}

    \item[{\sf op}]   ({\sf IN}) The semiring used in the matrix-matrix
    multiply.
    \scott{We could split this into two binary operators, map and reduce, one of which a monoid could be supplied.}

    \item[{\sf A}]    ({\sf IN}) The GraphBLAS matrix holding the values
    for the left-hand matrix in the multiplication.
    \scott{To used the transpose of a matrix, wrap this matrix in a transpose view by calling \code{grb::transpose(A)}.}

    \item[{\sf B}]    ({\sf IN}) The GraphBLAS matrix holding the values for
    the right-hand matrix in the multiplication.
    \scott{To used the transpose of a matrix, wrap this matrix in a transpose view by calling \code{grb::transpose(B)}.}

    \item[{\sf outp/replace\_flag}] ({\sf IN}) If a non-default mask (i.e. \code{grb::no_mask}) is specified,
    this flag controls what happens to the unmasked elements of the output.  If the flag is \code{true/grb::REPLACE}
    then the unmasked elements are cleared.  If the flag is \code{false/grb::MERGE}, the unmask elements are preserved in the final output. \\
\end{itemize}

\subparagraph{Type Requirements}

\begin{itemize}[leftmargin=1.1in]
    \item {\sf CMatrixType} must meet the requirements of a \textit{MatrixView}.  What are the basic matrix requirements?
    \item {\sf MaskType} must meet the requirements of a \emph{ConstMatrix} or (or is it \emph{Mask}: stored scalars must be convertible to bool) or \emph{MaskMatrixView}.  ComplementMatrixView or StructureMatrixView or StructuralComplementMatrixView, convertible to bool.  Range of index pairs denoting locations of the stored values)
    \item {\sf AMatrixType} and {\sf BMatrixType} must meet the requirements of a \emph{ConstMatrix} or \emph{ConstMatrixView}.
    \item {\sf AccumulatorType} must meet the requirements of a \emph{BinaryOperator}
    \item {\sf SemiringType} must meet the requirements of a \emph{GraphBLASSemiring}
\end{itemize}

If we decide to support breaking up the semiring into two parts without an identity then:

\begin{itemize}[leftmargin=1.1in]
    \item {\sf ReduceType} must meet the requirements of a \emph{CommutativeAssociativeBinaryOperator}
    \item {\sf MapType} must meet the requirements of a \emph{BinaryOperator}
\end{itemize}

Placeholder for named requirements:

\begin{itemize}
\item \emph{GraphBLASMatrix} - needs to satisfy the interface of a mutable GraphBLAS Matrix: nrows(), ncols(), nvals()
\item \emph{GraphBLASMaskMatrix} - A graphblas Matrix whose values are convertable to bool.
\item \emph{GraphBLASMatrixView} - Can provide a reference to a const GraphBLAS matrix
    \begin{itemize}
    \item \emph{GraphBLASMatrixComplementView}
    \item \emph{GraphBLASStructureMatrixView}
    \item \emph{GraphBLASStructuralComplementMatrixView}
    \item \emph{GraphBLASMatrixTransposeView}
    \end{itemize}
\item \emph{BinaryOperator}
\item \emph{GraphBLASSemiring} contains
    \begin{itemize}
    \item \emph{CommutativeAssociativeBinaryOperator}
    \item \emph{BinaryOperator}
    \end{itemize}
\end{itemize}

\begin{tabularx}{\textwidth}{X l}
Defined in header \texttt{<operations.hpp>}  &  \textbf{Notes} \\
\hline
\end{tabularx}

\paragraph{Return Values}

This function returns no values.  All errors result in exceptions being thrown.

\scott{There is a company that did not support exceptions and needed a return value semantic.  Do we change our philosophy to support.}

\paragraph{Exceptions}

\begin{itemize}[leftmargin=2.1in]
    \item[{\sf grb::panic\_error}]           (execution error, grb::runtime\_error) Unknown internal error.

    \item[{\sf grb::invalid\_object}] (execution error, runtime error) This is returned in any execution mode 
    whenever one of the opaque GraphBLAS objects (input or output) is in an invalid 
    state caused by a previous execution error.  Call {\sf GrB\_error()} to access 
    any error messages generated by the implementation.

    \item[{\sf grb::bad\_alloc}] (execution error, runtime error) Not enough memory available for the operation.
    \scott{is there a std::exception, std::bad\_alloc?}

    \item[{\sf grb::dimension\_mismatch}] (API error, logic error). Mask and/or matrix
    dimensions are incompatible. \scott{std::range\_error ?}
\end{itemize}

\paragraph{Blocking vs Non-Blocking Behaviour}

In blocking mode, the operation has completed successfully on return.
In non-blocking mode, this indicates that the compatibility 
tests on dimensions \scott{and domains for the input arguments passed successfully}. 
Either way, output matrix {\sf C} is ready to be used in the next method of
the sequence.

\paragraph{Description}

\scott{Does this specification need to the long duplicative mathematical descriptions
that are found in the C API Specification or can we refer to that spec here?}

\paragraph{Example}

\begin{minted}{c++}
        grb::Matrix A<float>({5, 10});
        grb::Matrix B<float>({5, 10});
        grb::Matrix C<float>({5, 5});
        grb::Matrix M<bool>({5, 5});
        // ...
        grb::mxm(C, M, grb::NoAccumulate(), 
                 grb::PlusTimesSemiring<float>(), A, B,
                 grb::MERGE);
        // ...
        // using all possible matrix views
        grb::mxm(C, complement(structure(M)), grb::NoAccumulate(), 
                 grb::PlusTimesSemiring<float>(), transpose(A), transpose(B),
                 grb::REPLACE);
\end{minted}

%-----------------------------------------------------------------------------
\section{{\sf grb::multiply}: Matrix-matrix multiply alternative form}

\paragraph{\syntax}

\begin{verbatim}
    // consider an additional form
    template<typename AMatrixType,
             typename BMatrixType,
             typename SemiringType,
             typename MaskType=grb::NoMask>
    auto multiply(AMatrixType const &A, BMatrixType const &B,
                  SemiringType semiring_op, MaskType const &Mask=MaskType());  // const& ?
\end{verbatim}

%-----------------------------------------------------------------------------
\section{{\sf mxv}: matrix-vector multiply}

\paragraph{\syntax}

\begin{minted}{c++}
    // overload using a semiring
    template<typename WVectorType,
             typename MaskType,
             typename AccumulatorType,
             typename SemiringType,
             typename AMatrixType,
             typename BMatrixType>
    void mxv(WVectorType            &w,
             MaskType         const &mask,
             AccumulatorType         accum,   // pass by value or const&?
             SemiringType            op,      // pass by value or const&?
             AMatrixType      const &A,
             UVectorType      const &u,
             OutputControlEnum       outp = MERGE);  //or bool replace_flag = false);

    // ...or...overload using two binary operators
    template<typename WVectorType,
             typename MaskType,
             typename AccumulatorType,
             typename ReduceType,
             typename MapType,
             typename AMatrixType,
             typename UVectorType>
    void mxv(WVectorType            &w,
             MaskType         const &mask,
             AccumulatorType         accum,   // pass by value or const&?
             ReduceType              reduce,  // pass by value or const&?
             MapType                 map,     // pass by value or const&?
             AMatrixType      const &A,
             UVectorType      const &u,
             OutputControlEnum       outp = MERGE);  //or bool replace_flag = false);
\end{minted}

%-----------------------------------------------------------------------------
\section{{\sf vxm}: vector-matrix multiply}

\paragraph{\syntax}

\begin{minted}{c++}
    // overload using a semiring
    template<typename WVectorType,
             typename MaskType,
             typename AccumulatorType,
             typename SemiringType,
             typename UVectorType,
             typename AMatrixType>
    void vxm(WVectorType            &w,
             MaskType         const &mask,
             AccumulatorType         accum,   // pass by value or const&?
             SemiringType            op,      // pass by value or const&?
             UVectorType      const &u,
             AMatrixType      const &A,
             OutputControlEnum       outp = MERGE);  //or bool replace_flag = false);

    // ...or...overload using two binary operators
    template<typename WVectorType,
             typename MaskType,
             typename AccumulatorType,
             typename ReduceType,
             typename MapType,
             typename UVectorType,
             typename AMatrixType>
    void vxm(WVectorType            &w,
             MaskType         const &mask,
             AccumulatorType         accum,   // pass by value or const&?
             ReduceType              reduce,  // pass by value or const&?
             MapType                 map,     // pass by value or const&?
             UVectorType      const &u,
             AMatrixType      const &A,
             OutputControlEnum       outp = MERGE);  //or bool replace_flag = false);
\end{minted}


%-----------------------------------------------------------------------------
\section{{\sf ewisemult}: element-wise multiplication, set intersection}

\paragraph{\syntax}

\begin{minted}{c++}
    // grb::vector overload
    template<typename T, typename I, typename Hint, typename Allocator,
             typename MaskType,
             typename AccumType,
             typename BinaryOpType,
             typename UVectorType,
             typename VVectorType>
    void ewisemult(vector<T, I, Hint, Allocator>     &w,
                   MaskType                    const &mask,
                   AccumType                   const &accum,
                   BinaryOpType                       op,
                   UVectorType                 const &u,
                   VVectorType                 const &v,
                   OutputControlEnum                  outp = MERGE);  // default val?

    // grb::matrix overload
    template<typename T, typename I, typename Hint, typename Allocator,
             typename MaskType,
             typename AccumType,
             typename BinaryOpType,
             typename AMatrixType,
             typename BMatrixType>
    void ewisemult(matrix<T, I, Hint, Allocator>     &C,
                   MaskT                       const &Mask,
                   AccumT                      const &accum,
                   BinaryOpT                          op,
                   AMatrixT                    const &A,
                   BMatrixT                    const &B,
                   OutputControlEnum                  outp = MERGE);  // default val?
\end{minted}


%-----------------------------------------------------------------------------
\section{{\sf ewiseadd}: element-wise addition}

\paragraph{\syntax}

\begin{minted}{c++}
    // grb::vector overload
    template<typename T, typename I, typename Hint, typename Allocator,
             typename MaskType,
             typename AccumType,
             typename BinaryOpType,
             typename UVectorType,
             typename VVectorType>
    void ewiseadd(vector<T, I, Hint, Allocator>     &w,
                  MaskType                    const &mask,
                  AccumType                   const &accum,
                  BinaryOpType                       op,
                  UVectorType                 const &u,
                  VVectorType                 const &v,
                  OutputControlEnum                  outp = MERGE);  // default val?

    // grb::matrix overload
    template<typename T, typename I, typename Hint, typename Allocator,
             typename MaskType,
             typename AccumType,
             typename BinaryOpType,
             typename AMatrixType,
             typename BMatrixType>
    void ewiseadd(matrix<T, I, Hint, Allocator>     &C,
                  MaskT                       const &Mask,
                  AccumT                      const &accum,
                  BinaryOpT                          op,
                  AMatrixT                    const &A,
                  BMatrixT                    const &B,
                  OutputControlEnum                  outp = MERGE);  // default val?
\end{minted}


%-----------------------------------------------------------------------------
%-----------------------------------------------------------------------------
\subsection{{\sf extract}: }

%-----------------------------------------------------------------------------
\subsection{{\sf extract}: standard vector variant}

\paragraph{\syntax}

\begin{minted}{c++}
    // standard grb::vector variant
    template<typename WVectorType,     // typename T, typename I, typename Hint, typename Allocator,
             typename MaskType,
             typename AccumType,
             typename UVectorType,
             typename SequenceType>
    void extract(WVectorType             &w,  // should we use vector<T, I, Hint, Allocator>?
                 MaskType          const &mask,
                 AccumType         const &accum,
                 UVectorType       const &u,
                 SequenceType      const &indices,
                 OutputControlEnum        outp = MERGE);  // default val?
\end{minted}

%-----------------------------------------------------------------------------
\subsection{{\sf extract}: standard matrix variant}

\paragraph{\syntax}

\begin{minted}{c++}
    // standard grb::matrix variant
    template<typename T, typename I, typename Hint, typename Allocator,
             typename MaskType,
             typename AccumType,
             typename AMatrixType,
             typename RowSequenceType,
             typename ColSequenceType>
    void extract(matrix<T, I, Hint, Allocator>  &C,
                 MaskT                    const &mask,
                 AccumT                   const &accum,
                 AMatrixT                 const &A,
                 RowSequenceT             const &row_indices,
                 ColSequenceT             const &col_indices,
                 OutputControlEnum               outp = MERGE);  // default val?
\end{minted}

%-----------------------------------------------------------------------------
\subsection{{\sf extract}: column (row) variant}

\paragraph{\syntax}

\begin{minted}{c++}
    // standard grb::matrix variant
    template<typename T, typename I, typename Hint, typename Allocator,
             typename MaskType,
             typename AccumType,
             typename AMatrixType,
             typename RowSequenceType>
    void extract(vector<T, I, Hint, Allocator>  &w,
                 MaskT                    const &mask,
                 AccumT                   const &accum,
                 AMatrixT                 const &A,
                 RowSequenceT             const &row_indices,
                 index_t                         col_index,
                 OutputControlEnum               outp = MERGE);  // default val?
\end{minted}


%-----------------------------------------------------------------------------
\section{{\sf assign}: }

%-----------------------------------------------------------------------------
\subsection{{\sf assign}: standard vector variant}

\paragraph{\syntax}

\begin{minted}{c++}
    template<typename T, typename I, typename Hint, typename Allocator,
             typename MaskT,
             typename AccumT,
             typename UVectorT,
             typename SequenceT,
             typename std::enable_if_t<is_vector_v<UVectorT>, int> = 0,
             typename ...WTags>
    void assign(vector<T, I, Hint, Allocator>      &w,
                MaskT                    const  &mask,
                AccumT                   const  &accum,
                UVectorT                 const  &u,
                SequenceT                const  &indices,
                OutputControlEnum                outp = MERGE)
\end{minted}

%-----------------------------------------------------------------------------
\subsection{{\sf assign}: standard matrix variant}

\paragraph{\syntax}

\begin{minted}{c++}
    template<typename CMatrixT, // it isn't enough to do T, I, Hint, Allocator
             typename MaskT,
             typename AccumT,
             typename AMatrixT,
             typename RowSequenceT,
             typename ColSequenceT,
             typename std::enable_if_t<is_matrix_v<AMatrixT>, int> = 0>
    inline void assign(CMatrixT              &C,
                       MaskT           const &Mask,
                       AccumT          const &accum,
                       AMatrixT        const &A,
                       RowSequenceT    const &row_indices,
                       ColSequenceT    const &col_indices,
                       OutputControlEnum      outp = MERGE)
\end{minted}


%-----------------------------------------------------------------------------
\subsection{{\sf assign}: column variant}

\paragraph{\syntax}

\begin{minted}{c++}
    template<typename CScalarT,
             typename MaskT,
             typename AccumT,
             typename UVectorT,
             typename SequenceT,
             typename ...CTags>
    inline void assign(Matrix<CScalarT, CTags...>  &C,
                       MaskT                 const &mask,  // a vector
                       AccumT                const &accum,
                       UVectorT              const &u,
                       SequenceT             const &row_indices,
                       IndexType                    col_index,
                       OutputControlEnum            outp = MERGE)
\end{minted}


%-----------------------------------------------------------------------------
\subsection{{\sf assign}: row variant}

\paragraph{\syntax}

\begin{minted}{c++}
    template<typename CScalarT,
             typename MaskT,
             typename AccumT,
             typename UVectorT,
             typename SequenceT,
             typename ...CTags>
    inline void assign(Matrix<CScalarT, CTags...>  &C,
                       MaskT                 const &mask,  // a vector
                       AccumT                const &accum,
                       UVectorT              const &u,
                       IndexType                    row_index,
                       SequenceT             const &col_indices,
                       OutputControlEnum            outp = MERGE)
\end{minted}


%-----------------------------------------------------------------------------
\subsection{{\sf assign}: vector constant variant}

\paragraph{\syntax}

\begin{minted}{c++}
    template<typename WVectorT,
             typename MaskT,
             typename AccumT,
             typename ValueT,
             typename SequenceT,
             typename std::enable_if<
                 std::is_convertible<ValueT,
                                     typename WVectorT::ScalarType>::value,
                 int>::type = 0>
    inline void assign(WVectorT                     &w,
                       MaskT                const   &mask,
                       AccumT               const   &accum,
                       ValueT                        val,
                       SequenceT            const   &indices,
                       OutputControlEnum             outp = MERGE)
\end{minted}


%-----------------------------------------------------------------------------
\subsection{{\sf assign}: matrix constant variant}

\paragraph{\syntax}

\begin{minted}{c++}
    template<typename CMatrixT,
             typename MaskT,
             typename AccumT,
             typename ValueT,
             typename RowSequenceT,
             typename ColSequenceT,
             typename std::enable_if<
                 std::is_convertible<ValueT,
                                     typename CMatrixT::ScalarType>::value,
                 int>::type = 0>
    inline void assign(CMatrixT             &C,
                       MaskT          const &Mask,
                       AccumT         const &accum,
                       ValueT                val,
                       RowSequenceT   const &row_indices,
                       ColSequenceT   const &col_indices,
                       OutputControlEnum     outp = MERGE)
\end{minted}



%-----------------------------------------------------------------------------
%-----------------------------------------------------------------------------
\subsection{{\sf apply}: }

\paragraph{\syntax}

\begin{minted}{c++}

\end{minted}


%-----------------------------------------------------------------------------
\subsection{{\sf select}: }

\paragraph{\syntax}

\begin{minted}{c++}

\end{minted}


%-----------------------------------------------------------------------------
\subsection{{\sf reduce}: }

\paragraph{\syntax}

\begin{minted}{c++}

\end{minted}


%-----------------------------------------------------------------------------
\subsection{{\sf transpose}: }

\paragraph{\syntax}

\begin{minted}{c++}

\end{minted}


%-----------------------------------------------------------------------------
\subsection{{\sf kronecker}: }

\paragraph{\syntax}

\begin{minted}{c++}

\end{minted}




%=============================================================================
%=============================================================================

\appendix
\chapter{Revision History}
\label{Chp:RevHistory}

Initial release 1.x.x


%--------------------------------------------------------------

\chapter{Examples}
\label{Chp:Examples}

\pagebreak
\nolinenumbers
\section{Example: level breadth-first search (BFS) in GraphBLAS}
{\scriptsize
%\lstinputlisting[language=C,numbers=left]{BFS5M.cpp}
}
\vfill

\pagebreak
\nolinenumbers
\section{Example: level BFS in GraphBLAS using apply}
{\scriptsize
%\lstinputlisting[language=C,numbers=left]{BFS6_apply.cpp}
}
\vfill

\pagebreak
\nolinenumbers
\section{Example: parent BFS in GraphBLAS}
{\scriptsize
%\lstinputlisting[language=C,numbers=left]{BFS7_parents.cpp}
}
\vfill

\pagebreak
\nolinenumbers
\section{Example: betweenness centrality (BC) in GraphBLAS}
\label{App:BCnobatch}
{\scriptsize
%\lstinputlisting[language=C,numbers=left]{BC1M_update.cpp}
}
\vfill

\pagebreak
\nolinenumbers
\section{Example: batched BC in GraphBLAS}
{\scriptsize
%\lstinputlisting[language=C,escapechar=|,numbers=left]{BC1_batch.cpp}
}
\vfill

\pagebreak
\nolinenumbers
\section{Example: maximal independent set (MIS) in GraphBLAS}
{\scriptsize
%\lstinputlisting[language=C,numbers=left]{MIS1.cpp}
}
\vfill

\pagebreak
\nolinenumbers
\section{Example: counting triangles in GraphBLAS}
{\scriptsize
%\lstinputlisting[language=C,numbers=left]{TC1.cpp}
}
\vfill
\pagebreak


%\def\IEEEbibitemsep{3pt plus .5pt}
%\bibliographystyle{IEEEtran}
%\bibliography{refs}

\end{document}
