%-----------------------------------------------------------------------------
\section{GraphBLAS Operations}

%-----------------------------------------------------------------------------
\subsection{{\sf mxm}: Matrix-matrix multiply}

Multiplies a matrix with another matrix on a semiring. The result is stored in a matrix.

\paragraph{\syntax}

\begin{verbatim}
    template<typename CMatrixT,
             typename MaskT,
             typename AccumT,
             typename SemiringT,
             typename AMatrixT,
             typename BMatrixT>
    void mxm(CMatrixT         &C,
             MaskT      const &Mask,
             AccumT     const &accum,
             SemiringT         op,
             AMatrixT   const &A,
             BMatrixT   const &B,
             OutputControlEnum outp = MERGE)
\end{verbatim}

\paragraph{Template Parameters}

\begin{itemize}[leftmargin=1.1in]
    \item[{\sf CMatrixT}]    The type of the output Matrix (Matrix concept)
    \item[{\sf MaskT}]       The type of the write mask Matrix (Matrix concept)
    \item[{\sf AccumT}]      The type of the accumulation binary operator
    \item[{\sf SemiringT}]   The type of the semiring to use for matrix multiplication \scott{TODO: should this be separate map and reduce operators}
    \item[{\sf AMatrixT}]    The type of the left-hand input Matrix (Matrix concept)
    \item[{\sf BMatrixT}]    The type of the right-hand input Matrix (Matrix concept)
\end{itemize}

\paragraph{Parameters}

\begin{itemize}[leftmargin=1.1in]
    \item[{\sf C}]    ({\sf INOUT}) An existing GraphBLAS matrix. On input,
    the matrix provides values that may be accumulated with the result of the
    matrix product.  On output, the matrix holds the results of the
    operation.

    \item[{\sf Mask}] ({\sf IN}) An optional ``write'' mask that controls which
    results from this operation are stored into the output matrix {\sf C}. The 
    mask dimensions must match those of the matrix {\sf C}. If the 
    {\sf GrB\_STRUCTURE} descriptor is {\em not} set for the mask, the domain 
    of the {\sf Mask} matrix must be of type {\sf bool} or any of the predefined 
    ``built-in'' types in Table~\ref{Tab:PredefinedTypes}.  If the default
    mask is desired (\ie, a mask that is all {\sf true} with the dimensions of {\sf C}), 
    {\sf GrB\_NULL} should be specified. \scott{STRUCTURE\_ONLY changes.}

    \item[{\sf accum}] ({\sf IN}) An optional binary operator used for accumulating
    entries into existing {\sf C} entries.
    %: ${\sf accum} = \langle \bDout({\sf accum}),\bDin1({\sf accum}),
    %\bDin2({\sf accum}), \odot \rangle$. 
    If assignment rather than accumulation is
    desired, {\sf GrB\_NULL} should be specified.

    \item[{\sf op}]   ({\sf IN}) The semiring used in the matrix-matrix
    multiply.
    %: ${\sf op}=\langle \bDout({\sf op}),\bDin1({\sf op}),\bDin2({\sf op}),\oplus,\otimes,0 \rangle$.

    \item[{\sf A}]    ({\sf IN}) The GraphBLAS matrix holding the values
    for the left-hand matrix in the multiplication.

    \item[{\sf B}]    ({\sf IN}) The GraphBLAS matrix holding the values for
    the right-hand matrix in the multiplication.

    \item[{\sf desc}] ({\sf IN}) An optional operation descriptor. If
    a \emph{default} descriptor is desired, {\sf GrB\_NULL} should be
    specified. Non-default field/value pairs are listed as follows:  \\
    
    \hspace*{-2em}\begin{tabular}{lllp{2.7in}}
        Param & Field  & Value & Description \\
        \hline
        {\sf C}    & {\sf GrB\_OUTP} & {\sf GrB\_REPLACE} & Output matrix {\sf C}
        is cleared (all elements removed) before the result is stored in it.\\

        {\sf Mask} & {\sf GrB\_MASK} & {\sf GrB\_STRUCTURE}   & The write mask is
        constructed from the structure (pattern of stored values) of the input
        {\sf Mask} matrix. The stored values are not examined.\\

        {\sf Mask} & {\sf GrB\_MASK} & {\sf GrB\_COMP}   & Use the
        complement of {\sf Mask}. \\

        {\sf A}    & {\sf GrB\_INP0} & {\sf GrB\_TRAN}   & Use transpose of {\sf A}
        for the operation. \\

        {\sf B}    & {\sf GrB\_INP1} & {\sf GrB\_TRAN}   & Use transpose of {\sf B}
        for the operation. \\
    \end{tabular}
\end{itemize}

\paragraph{Return Values}

\begin{itemize}[leftmargin=2.1in]
    \item[{\sf GrB\_SUCCESS}]         In blocking mode, the operation completed
    successfully. In non-blocking mode, this indicates that the compatibility 
    tests on dimensions and domains for the input arguments passed successfully. 
    Either way, output matrix {\sf C} is ready to be used in the next method of
    the sequence.

    \item[{\sf GrB\_PANIC}]           Unknown internal error.

    \item[{\sf GrB\_INVALID\_OBJECT}] This is returned in any execution mode 
    whenever one of the opaque GraphBLAS objects (input or output) is in an invalid 
    state caused by a previous execution error.  Call {\sf GrB\_error()} to access 
    any error messages generated by the implementation.

    \item[{\sf GrB\_OUT\_OF\_MEMORY}] Not enough memory available for the operation.

    \item[{\sf GrB\_UNINITIALIZED\_OBJECT}] One or more of the GraphBLAS objects 
    has not been initialized by a call to {\sf new} (or {\sf Matrix\_dup} for matrix
    parameters).

    \item[{\sf GrB\_DIMENSION\_MISMATCH}] Mask and/or matrix
    dimensions are incompatible.

    \item[{\sf GrB\_DOMAIN\_MISMATCH}]    The domains of the various matrices are
    incompatible with the corresponding domains of the semiring or
    accumulation operator, or the mask's domain is not compatible with {\sf bool}
    (in the case where {\sf desc[GrB\_MASK].GrB\_STRUCTURE} is not set).\scott{STRUCTURE\_ONLY changes.}
\end{itemize}

\paragraph{Description}

%-----------------------------------------------------------------------------
\subsection{{\sf mxv}: Matrix-vector multiply}

\subsection{{\sf vxm}: Vector-matrix multiply}


%-----------------------------------------------------------------------------
\subsection{ops\_ewisemult\_ewiseadd}

%-----------------------------------------------------------------------------
\subsection{ops\_extract}

%-----------------------------------------------------------------------------
\subsection{ops\_assign} 

%-----------------------------------------------------------------------------
\subsection{ops\_apply}

%-----------------------------------------------------------------------------
\subsection{ops\_reduce\_transpose}

%-----------------------------------------------------------------------------
\subsection{ops\_kronecker}
