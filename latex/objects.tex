\chapter{Objects}
\label{Chp:Objects}


% ========================================================================
\section{Operators}

\scott{All callable objects must be supported, including <functional> and lambdas.}

% ========================================================================
\section{Monoids}

Do we define the concept of a Monoid here.  We don't "new" them, rather they
are passed as templated arguments to the backend implementation.


\scott{Likely only structs, with operator() + identity()? Do we support result\_type?}


% ========================================================================
\section{Semirings}

Do we define the concept of a Semiring here.  We don't "new" them, rather they
are passed as templated arguments to the backend implementation.


\scott{likely only structs with add90, mult(), and zero().  Do we support 
first/second\_argument\_type and result\_type?}

% ========================================================================
\section{Sequences}
\label{Sec:Sequences}

{\sf GrB\_ALL} (AllIndices class in GBTL) is an example of a compact sequence.

There should be other sequences that replace "arrays of indices" that can also be substituted.

This is also important for distributed implementations where enumerating all indices in a range can be too large.

% ========================================================================
\section{{\sf GrB\_NULL} for Mask and Accumulator}
\label{Sec:NoMaskNoAccum}

Do we want a single {\sf GrB\_NULL} type that can be passed as the mask or accumulator 
to indicate "no mask" and "no accumulate", respectively.  GBTL currently has two
different empty classes N={\sf NoMask} and {\sf NoAccumulate}

% ========================================================================
\section{Vectors}
\label{Sec:Vectors}

\scott{Add a discussion of template parameters}
Template arguments:
\begin{itemize}[leftmargin=1.1in]
\item[ScalarT]  the scalar/value/domain type stored as values in the vector (i.e. the domain)
\item[IndexT]   the unsigned integer type used to specify a location in the vector.
\item[...TagsT] variadic (variable number of type params) to be unpacked by the implementation.
Implementation "hints" or "suggestions" like sparse vs dense.
\end{itemize}

% ========================================================================
\section{Matrices}
\label{Sec:Matrices}

Definition:

\begin{verbatim}
template <typename T,
          typename I = std::size_t,
          typename Hint = grb::sparse,
          typename Allocator = std::allocator<T>>
class matrix;
\end{verbatim}

\subsection{Member Types}

\begin{tabularx}{\textwidth}{l X}
\textbf{Member Type} & \textbf{Definition}\\
\hline
\texttt{value\_type} & \texttt{T} \\
\hline
\texttt{index\_type} & \texttt{I} \\
\hline
\texttt{size\_type} & Unsigned integer type (usually \texttt{std::size\_t})\\
\hline
\texttt{difference\_type} & Signed integer type (usually \texttt{std::ptrdiff\_t})\\
\hline
\texttt{hint\_type} & \texttt{Hint}\\
\hline
\texttt{allocator\_type} & \texttt{Allocator}\\
\hline
\texttt{reference} & \textit{MatrixReference} to \texttt{value\_type}\\
\hline
\texttt{const\_reference} & \textit{MatrixReference} to \texttt{const value\_type}\\
\hline
\texttt{pointer} & \textit{MatrixIterator} to \texttt{value\_type}\\
\hline
\texttt{const\_pointer} & \textit{MatrixIterator} to \texttt{const value\_type}\\
\hline
\texttt{iterator} & \textit{MatrixIterator} to \texttt{value\_type}\\
\hline
\texttt{const\_iterator} & \textit{Matrixiterator} to \texttt{const value\_type}\\
\end{tabularx}

\subsection{Member Functions}

\begin{tabularx}{\textwidth}{l X}
\textbf{Constructors}\\
\hline
\texttt{matrix()} & Construct empty matrix of dimension $0 \times 0$\\
\hline
\texttt{matrix(index\_t dim)} & Construct matrix of dimension $\texttt{dim[0]} \times \texttt{dim[1]}$\\
\hline
\texttt{matrix(std::initializer\_list<size\_type> dim)} & Construct matrix of dimension $\texttt{*dim.begin()} \times \texttt{*(++dim.begin)}$\\
\hline
\texttt{matrix(std::string fname)} & Construct matrix object from Matrix Market file at filepath \texttt{fname}.\\
\hline
\texttt{matrix(const matrix\& other)} & Copy construct matrix from \texttt{other}.\\
\hline
\texttt{matrix(matrix\&\& other)} & Move construct matrix from \texttt{other}.\\
\hline
\textbf{Iterators}\\
\hline
\texttt{iterator begin()} & Non-const iterator to the beginning\\
\hline
\texttt{const\_iterator begin() const} & Const iterator to the beginning\\
\hline
\texttt{iterator end()} & Non-const iterator to the end\\
\hline
\texttt{const\_iterator end() const} & Const iterator to the end\\
\hline
\textbf{Capacity and Dimensions}\\
\hline
\texttt{size\_type size() const} & Number of stored values\\
\hline
\texttt{index\_t shape() const} & Dimensions of matrix\\
\hline
\textbf{Lookup}\\
\texttt{reference at(index\_t index)} & Access specific element with bounds checking\\
\hline
\texttt{const\_reference at(index\_t index) const} & Access specific element with bounds checking\\
\hline
\texttt{reference operator[](index\_t index)} & Access or insert a specified element\\
\hline
\texttt{const\_reference operator[](index\_t index) const} & Access a specified element\\
\hline
\texttt{iterator find(index\_t index)} & Find element at a specific index\\
\hline
\texttt{const\_iterator find(index\_t index) const} & Find element at a specific index\\
\end{tabularx}

% ========================================================================
\section{Masks}
\label{Sec:Masks}

Do we support void Scalar type for matrices and vectors?

% ========================================================================
\section{Descriptors}
\label{Sec:Descriptors}

\scott{There will be no descriptors...so let it be written...so let it be done.}
