\chapter{Classes and Methods}
\label{Chp:Methods}

This chapter defines the behavior of all the methods in the GraphBLAS C API.
All methods can be declared for use in programs by including the {\tt GraphBLAS.h} header file.

We would like to emphasize that no GraphBLAS method will imply a predefined order over any associative operators. Implementations of the GraphBLAS are encouraged to exploit associativity to optimize performance of any GraphBLAS method. This holds even if the definition of the GraphBLAS method implies a fixed order for the associative operations.

%-----------------------------------------------------------------------------
\section{Context Methods}

%-----------------------------------------------------------------------------
\section{Object Classes and Methods}

This section describes methods that setup and operate on GraphBLAS opaque objects
but are not part of the the GraphBLAS math specification.  \scott{Does this need to be said?  Is it accurate?}

%-----------------------------------------------------------------------------
\subsection{Algebra Classes}

\scott{There are no separate "Types".  UnaryOp and BinaryOps are all callable objects.  Monoids and Semirings are class templates}

We could further define the concept and give a few examples of functions, functors and lambdas  that can be used where a unaryop, binaryop, monoid and semiring are needed.

Do we provide and list all of the predefined structs?

subsubsection{


%-----------------------------------------------------------------------------
\subsection{Vector Class}

This section defines the minimum required public interface to the "API" (I refer to it
as the frontend) Vector class.  \scott{Are implementers free to add to the public API?
I think it is possible it may be necessary?}

\subsubsection{Template and Type Definitions}

No subclassing in API

\begin{verbatim}
    template<typename ScalarT, typename IndexT, typename... TagsT>
    class Vector
    {
    public:
        typedef ScalarT ScalarType;
        typedef IndexT  IndexType;

        ...
    };
\end{verbatim}

We need to defined the template parameters:
\begin{itemize}[leftmargin=1.1in]
\item[\sf ScalarT]  The type of the elements.  Must they meet the requirements of CopyAssignable and CopyConstructible (pre C++11) or something else.  \scott{Here is the statement from the C++ Standard Graph Library proposal: "The graph value type defined by the user.  It can be most valid C++ value type including class, struct, tuple, union, enum, array, reference or scalar value. If no value is needed then the empty\_value struct can be used."}
\item[\sf IndexT] this type is used for a vertex’s index. \scott{Should we restrict to unsigned integer types?}
\item[\sf TagsT] variadic arguments TBD
\end{itemize}

What are the possible types sent through TagsT?

The class needs to define a few typedefs
\begin{itemize}[leftmargin=1.1in]
\item[\sf ScalarType] Echos the ScalarT template argument used
\item[\sf IndexType]  Echos the IndexT template argument used
\end{itemize}

\scott{class is a base class}
\scott{default constructor deleted}
\scott{destructor not virtual, do we allow implicit destructor?}
\scott{"dup" functionality: copy ctor and assignment operator both? What form?}
\scott{Do we require move semantics}


\subsubsection{Standard Constructors}

\paragraph{\syntax}

\begin{verbatim}
    Vector() = delete;         // default construction not supported
    Vector(Vector const &rsh); // Should we specify copy constructor?
    Vector(Vector &&rhs);      // Should we specify move constructor?
\end{verbatim}

\paragraph{Description}

\scott{How should we document the various constructors}

Default construction is not supported; i.e., a size must be specified.

Copying and moving vectors of exactly the same type could be supported.  
Copy would be O(nsize) complexity, move could be O(1).


\subsubsection{Constructor: create a new Vector}

\paragraph{\syntax}

\begin{verbatim}
    Vector(IndexType nsize);   // Standard constructor
\end{verbatim}

\paragraph{Parameters}

\begin{itemize}[leftmargin=1.1in]
    \item[{\sf nsize}] ({\sf IN}) The initial size of the vector being created.
\end{itemize}

\paragraph{Exceptions}

\begin{itemize}[leftmargin=2.1in]
    \item[{\sf grb::PanicException}]   Unknown internal error.
    \item[{\sf std::bad\_alloc}]        Not enough memory available for operation.
    \item[{\sf std::invalid\_argument}] {\sf nsize} is zero.
\end{itemize}

\paragraph{Description}

Creates a new vector $\vector{v}$ that can store {\sf ScalarT} type elements, with size {\sf nsize}, 
and empty $\mathbf{L}(\vector{v})$.

\scott{should we leverage as many STL exceptions classes as possible?}

%-----------------------
\subsubsection{Destructor}

\paragraph{\syntax}

\begin{verbatim}
    ~Vector();                  // not virtual (i.e. no subclassing)
\end{verbatim}

\paragraph{Description}

Releases all resources held by this container including all implementation data structures.
The destructor is not virtual as this class is not intended to participate in runtime polymorphism.

%-----------------------
\subsubsection{Assignment Operator}

\paragraph{\syntax}

\begin{verbatim}
    Vector& operator=(Vector const &rhs)
\end{verbatim}

\paragraph{Parameters}

\begin{itemize}[leftmargin=1.1in]
    \item[{\sf rhs}] ({\sf IN}) The vector to copy from (was constructed with the same template arguments).
\end{itemize}

\paragraph{Exceptions}

\begin{itemize}[leftmargin=2.1in]
    \item[{\sf grb::PanicException}]   Unknown internal error.
    \item[{\sf std::bad\_alloc}]        Not enough memory available for operation.
\end{itemize}

\paragraph{Description}

A copy is made of {\sf rsh} into this vector.  This vector changes size and content to match {\sf rhs}.
On return the state of {\sf rhs} is unchanged.

\scott{Should we support copy assignment?}  


%-----------------------
\subsubsection{Move operator}

\paragraph{\syntax}

\begin{verbatim}
    Vector& operator=(Vector const &&rhs)
\end{verbatim}

\paragraph{Exceptions}

\begin{itemize}[leftmargin=2.1in]
    \item[{\sf grb::PanicException}]   Unknown internal error.
\end{itemize}

\paragraph{Description}

The contents of {\sf rhs} moved into this vector.  This vector changes size and content to match {\sf rhs}.
On return the state of {\sf rhs} is "indeterminate".

\scott{Should we support move assignment?}
\scott{is this noexcept?}

%-----------------------
\subsubsection{{\sf build} Method: from iterators}

\scott{How many different overloads should be supported?}

\paragraph{\syntax}

\begin{verbatim}
    // build methods
    template<typename RAIteratorI,
             typename RAIteratorV,
             typename BinaryOpT = GraphBLAS::Second<ScalarT> >
    void build(RAIteratorI  i_it,
               RAIteratorV  v_it,
               IndexT       nvals,   // size_t?
               BinaryOpT    dup = BinaryOpT());
\end{verbatim}

\paragraph{Template Parameters}

\begin{itemize}[leftmargin=1.1in]
    \item[{\sf RAIteratorI}] A random access iterator over IndexT objects.  (type traits needed to enforce?)
    \item[{\sf RAIteratorV}] A random access iterator over ScalarT objects.  (type traits needed to enforce?)
    \item[{\sf BinarOpT}]    A callable in the form of a binary function
                             that takes two ScalarT items and returns a ScalarT. (type traits needed to enforce)
\end{itemize}

\paragraph{Parameters}

\begin{itemize}[leftmargin=1.1in]
    \item[{\sf i\_it}]   ({\sf IN}) OUT? Iterator over a sequence of indices
    \item[{\sf v\_it}]   ({\sf IN}) OUT? Iterator over a sequence of values
    \item[{\sf nvals}]   ({\sf IN}) The number of entries that will be iterated over from each sequence
    \item[{\sf dup}]     ({\sf IN}) An associative and commutative binary operator
                                    to apply when duplicate values for the same
                                    location (index) are present in the input sequences.
\end{itemize}

\paragraph{Exceptions}

\begin{itemize}[leftmargin=2.1in]
    \item[{\sf grb::PanicException}]   Unknown internal error.
    \item[{\sf std::bad\_alloc}]       Not enough memory available for operation.
    \item[{\sf std::out\_of\_range}]   A value in the index sequence is outside the
                                       allowed range for the vector.
    \item[{\sf grb::OutputNotEmpty}]   Vector is not empty when this is called.                                  
\end{itemize}

\paragraph{Description}

%-----------------------
\subsubsection{{\sf build} Method: from containers.}

\scott{How many different overloads should be supported?}

\paragraph{\syntax}

\begin{verbatim}
    template<typename BinaryOpT = GraphBLAS::Second<ScalarType> >
    inline void build(std::vector<IndexT>     const &indices,
                      std::vector<ScalarT>    const &values,
                      BinaryOpT                      dup = BinaryOpT());
\end{verbatim}

\paragraph{Template Parameters}

\begin{itemize}[leftmargin=1.1in]
    \item[{\sf BinarOpT}]    A callable in the form of a binary function
                             that takes two ScalarT items and returns a ScalarT. (type traits needed to enforce)
\end{itemize}

\paragraph{Parameters}

\begin{itemize}[leftmargin=1.1in]
    \item[{\sf indices}]  ({\sf IN}) container of indices
    \item[{\sf values}]   ({\sf IN}) container of values
    \item[{\sf dup}]      ({\sf IN}) An associative and commutative binary operator
                                     to apply when duplicate values for the same
                                     location (index) are present in the input sequences.
\end{itemize}

\paragraph{Exceptions}

\begin{itemize}[leftmargin=2.1in]
    \item[{\sf grb::PanicException}]   Unknown internal error.
    \item[{\sf grb::BadSize}]          indices and values containers do not have
                                       the same number of elements (or should we 
                                       also provide nvals? In which case, BadSize 
                                       means not enough values in containers).
    \item[{\sf std::bad\_alloc}]       Not enough memory available for operation.
    \item[{\sf std::out\_of\_range}]   A value in the index sequence is outside the
                                       allowed range for the vector.
    \item[{\sf grb::OutputNotEmpty}]   Vector is not empty when this is called.                                  
\end{itemize}

\paragraph{Description}

%-----------------------
\subsubsection{{\sf clear} Method}
\begin{verbatim}
    void clear(); // noexcept?
\end{verbatim}

%-----------------------
\subsubsection{{\sf size} Method}
\begin{verbatim}
    IndexType size() const noexcept;
\end{verbatim}

%-----------------------
\subsubsection{{\sf nvals} Method}
\begin{verbatim}
    IndexType nvals() const noexcept;
\end{verbatim}

%-----------------------
\subsubsection{{\sf resize} Method}
\begin{verbatim}
    void resize(IndexType new_size);
\end{verbatim}


%-----------------------
\subsubsection{{\sf hasElement} Method}
\begin{verbatim}
    bool hasElement(IndexType index) const;
\end{verbatim}

%-----------------------
\subsubsection{{\sf setElement} Method}
\begin{verbatim}
    void setElement(IndexType index, ScalarT const &new_val);
\end{verbatim}

%-----------------------
\subsubsection{{\sf removeElement} Method}
\begin{verbatim}
    void removeElement(IndexType index);
\end{verbatim}

%-----------------------
\subsubsection{{\sf extractElement} Method}
\begin{verbatim}
    ScalarT extractElement(IndexType index) const;
\end{verbatim}


%-----------------------
\subsubsection{{\sf extractTuples} Method}
\begin{verbatim}
    template<typename RAIteratorIT,
             typename RAIteratorVT>
    void extractTuples(RAIteratorIT        i_it,
                       RAIteratorVT        v_it) const;

    void extractTuples(IndexArrayType        &indices,
                       std::vector<ScalarT>  &values) const;
\end{verbatim}

%-----------------------
\subsubsection{Iterators or ranges?}

\scott{Do we require iterators to iterate over stored elements?}


%-----------------------------------------------------------------------------
%-----------------------------------------------------------------------------
\subsection{Matrix Class}

This section defines the minimum required public interface to the "API" (I refer to it
as the frontend) Matrix class.  \scott{Are implementers free to add to the public API?
I think it is possible it may be necessary?}

%-----------------------------------------------------------------------------
\subsection{Descriptor Class...There isn't one}

\scott{There should be no Descriptors...I hope...maybe Views}.

%-----------------------------------------------------------------------------
%-----------------------------------------------------------------------------
\subsection{Views}

\subsubsection{transpose method, TransposeView class}

Only for matrices.  Do we need to specify the TransposeView class?
Is the class an implementation detail?  Can we specify the method without stating its return type?


\subsubsection{structure method, StructureView class}

Only for masks.  Do we need to specify the StructureView class?
Is the class an implementation detail?  Can we specify the method without stating its return type?


\subsubsection{complement method, ComplementView class}

Only for masks.  Do we need to specify the ComplementView class?
Is the class an implementation detail?  Can we specify the method without stating its return type?


\subsubsection{Composing mask views}

Only complement(structure(mask)) is valid  (structure(complement(mask) is not valid).

Do we talk about StructuralComplementView class?

Is the class an implementation detail?  Can we specify the method without stating its return type?

%-----------------------------------------------------------------------------
\subsection{{\sf free} Methods}

\scott{should be replaced with class destructors}.

%-----------------------------------------------------------------------------
\subsection{{\sf wait} Methods}

\scott{should be moved to a class method in each object type.}

%-----------------------------------------------------------------------------
\subsection{{\sf error} Methods}

\scott{should be moved to a class method in each object type.}

%-----------------------------------------------------------------------------
%-----------------------------------------------------------------------------
% \mintinline{cpp}{code}
%=============================================================================
\chapter{GraphBLAS Operations}
\label{Ch:Operations}

%-----------------------------------------------------------------------------
\section{{\sf grb::mxm}: matrix-matrix multiply}

\paragraph{\syntax}

\begin{minted}{c++}
    template<typename CMatrixType,
             typename MaskType,
             typename AccumulatorType,
             typename SemiringType,
             typename AMatrixType,
             typename BMatrixType>
    void mxm(CMatrixType            &C,
             MaskType         const &Mask,
             AccumulatorType         accum,   // pass by value or const&?
             SemiringType            op,      // pass by value or const&?
             AMatrixType      const &A,
             BMatrixType      const &B,
             OutputControlEnum       outp = MERGE);  //or bool replace_flag = false);

    // ...or...
    template<typename CMatrixType,
             typename MaskType,
             typename AccumulatorType,
             typename ReduceType,
             typename MapType
             typename AMatrixType,
             typename BMatrixType>
    void mxm(CMatrixType            &C,
             MaskType         const &Mask,
             AccumulatorType         accum,   // pass by value or const&?
             ReduceType              reduce,  // pass by value or const&?
             MapType                 map,     // pass by value or const&?
             AMatrixType      const &A,
             BMatrixType      const &B,
             OutputControlEnum       outp = MERGE);  //or bool replace_flag = false);
\end{minted}

Multiplies two GraphBLAS matrices using the operators and identity defined by a GraphBLAS semiring. An optional accumulator and write mask can also be specified. The result is stored in third GraphBLAS matrix.

\begin{enumerate}
\item Any notes go here.
\end{enumerate}

\paragraph{Parameters}

\begin{itemize}[leftmargin=1.1in]
    \item[{\sf C}]    ({\sf INOUT}) A GraphBLAS \emph{Matrix} type. On input,
    the matrix provides values that may be accumulated with the result of the
    matrix product.  On output, the matrix holds the results of the
    operation.

    \item[{\sf Mask}] ({\sf IN}) An optional ``write'' mask (a \emph{ConstMatrixView}) that controls which
    results from this operation are stored into the output matrix {\sf C}. The 
    mask dimensions must match those of the matrix {\sf C}. 
    \scott{
    To complement/invert the logic of a mask, wrap the mask in a complement view by calling \code{grb::complement(Mask)}.
    To use the structure of this matrix only, wrap the mask in a structure view by calling \code{grb::structure(Mask)}.
    These views can be compose to get the complement of the structure of a mask by nesting these calls:
    \code{grb::complement(grb::structure(Mask))} (Note that \code{grb::structure(grb::complement(Mask))} is invalid and should not compile).
    If it is not wrapped in the \code{grb::structure} view, the domain 
    of the {\sf Mask} matrix must be of type that can be compared to \code{bool}.}
    If the default
    mask is desired (\ie, logically, a mask that is all {\sf true} with the dimensions of {\sf C}), 
    {\sf grb::no\_mask} should be passed in for this argument.
    \scott{What should be passed?  \code{grb::no_mask} or \code{grb::NoMask()}}

    \item[{\sf accum}] ({\sf IN}) An optional binary operator used for accumulating
    entries into existing {\sf C} entries.  If assignment rather than accumulation is
    desired, \code{grb::no_accum} should be specified. \scott{What should be passed?
    \code{grb::no_accum} or \code{grb::NoAccumulate()}}

    \item[{\sf op}]   ({\sf IN}) The semiring used in the matrix-matrix
    multiply.
    \scott{We could split this into two binary operators, map and reduce, one of which a monoid could be supplied.}

    \item[{\sf A}]    ({\sf IN}) The GraphBLAS matrix holding the values
    for the left-hand matrix in the multiplication.
    \scott{To used the transpose of a matrix, wrap this matrix in a transpose view by calling \code{grb::transpose(A)}.}

    \item[{\sf B}]    ({\sf IN}) The GraphBLAS matrix holding the values for
    the right-hand matrix in the multiplication.
    \scott{To used the transpose of a matrix, wrap this matrix in a transpose view by calling \code{grb::transpose(B)}.}

    \item[{\sf outp/replace\_flag}] ({\sf IN}) If a non-default mask (i.e. \code{grb::no_mask}) is specified,
    this flag controls what happens to the unmasked elements of the output.  If the flag is \code{true/grb::REPLACE}
    then the unmasked elements are cleared.  If the flag is \code{false/grb::MERGE}, the unmask elements are preserved in the final output. \\
\end{itemize}

\subparagraph{Type Requirements}

\begin{itemize}[leftmargin=1.1in]
    \item {\sf CMatrixType} must meet the requirements of a \textit{MatrixView}.  What are the basic matrix requirements?
    \item {\sf MaskType} must meet the requirements of a \emph{ConstMatrix} or (or is it \emph{Mask}: stored scalars must be convertible to bool) or \emph{MaskMatrixView}.  ComplementMatrixView or StructureMatrixView or StructuralComplementMatrixView, convertible to bool.  Range of index pairs denoting locations of the stored values)
    \item {\sf AMatrixType} and {\sf BMatrixType} must meet the requirements of a \emph{ConstMatrix} or \emph{ConstMatrixView}.
    \item {\sf AccumulatorType} must meet the requirements of a \emph{BinaryOperator}
    \item {\sf SemiringType} must meet the requirements of a \emph{GraphBLASSemiring}
\end{itemize}

If we decide to support breaking up the semiring into two parts without an identity then:

\begin{itemize}[leftmargin=1.1in]
    \item {\sf ReduceType} must meet the requirements of a \emph{CommutativeAssociativeBinaryOperator}
    \item {\sf MapType} must meet the requirements of a \emph{BinaryOperator}
\end{itemize}

Placeholder for named requirements:

\begin{itemize}
\item \emph{GraphBLASMatrix} - needs to satisfy the interface of a mutable GraphBLAS Matrix: nrows(), ncols(), nvals()
\item \emph{GraphBLASMaskMatrix} - A graphblas Matrix whose values are convertable to bool.
\item \emph{GraphBLASMatrixView} - Can provide a reference to a const GraphBLAS matrix
    \begin{itemize}
    \item \emph{GraphBLASMatrixComplementView}
    \item \emph{GraphBLASStructureMatrixView}
    \item \emph{GraphBLASStructuralComplementMatrixView}
    \item \emph{GraphBLASMatrixTransposeView}
    \end{itemize}
\item \emph{BinaryOperator}
\item \emph{GraphBLASSemiring} contains
    \begin{itemize}
    \item \emph{CommutativeAssociativeBinaryOperator}
    \item \emph{BinaryOperator}
    \end{itemize}
\end{itemize}

\begin{tabularx}{\textwidth}{X l}
Defined in header \texttt{<operations.hpp>}  &  \textbf{Notes} \\
\hline
\end{tabularx}

\paragraph{Return Values}

This function returns no values.  All errors result in exceptions being thrown.

\scott{There is a company that did not support exceptions and needed a return value semantic.  Do we change our philosophy to support.}

\paragraph{Exceptions}

\begin{itemize}[leftmargin=2.1in]
    \item[{\sf grb::panic\_error}]           (execution error, grb::runtime\_error) Unknown internal error.

    \item[{\sf grb::invalid\_object}] (execution error, runtime error) This is returned in any execution mode 
    whenever one of the opaque GraphBLAS objects (input or output) is in an invalid 
    state caused by a previous execution error.  Call {\sf GrB\_error()} to access 
    any error messages generated by the implementation.

    \item[{\sf grb::bad\_alloc}] (execution error, runtime error) Not enough memory available for the operation.
    \scott{is there a std::exception, std::bad\_alloc?}

    \item[{\sf grb::dimension\_mismatch}] (API error, logic error). Mask and/or matrix
    dimensions are incompatible. \scott{std::range\_error ?}
\end{itemize}

\paragraph{Blocking vs Non-Blocking Behaviour}

In blocking mode, the operation has completed successfully on return.
In non-blocking mode, this indicates that the compatibility 
tests on dimensions \scott{and domains for the input arguments passed successfully}. 
Either way, output matrix {\sf C} is ready to be used in the next method of
the sequence.

\paragraph{Description}

\scott{Does this specification need to the long duplicative mathematical descriptions
that are found in the C API Specification or can we refer to that spec here?}

\paragraph{Example}

\begin{minted}{c++}
        grb::Matrix A<float>({5, 10});
        grb::Matrix B<float>({5, 10});
        grb::Matrix C<float>({5, 5});
        grb::Matrix M<bool>({5, 5});
        // ...
        grb::mxm(C, M, grb::NoAccumulate(), 
                 grb::PlusTimesSemiring<float>(), A, B,
                 grb::MERGE);
        // ...
        // using all possible matrix views
        grb::mxm(C, complement(structure(M)), grb::NoAccumulate(), 
                 grb::PlusTimesSemiring<float>(), transpose(A), transpose(B),
                 grb::REPLACE);
\end{minted}

%-----------------------------------------------------------------------------
\section{{\sf grb::multiply}: Matrix-matrix multiply alternative form}

\paragraph{\syntax}

\begin{verbatim}
    // consider an additional form
    template<typename AMatrixType,
             typename BMatrixType,
             typename SemiringType,
             typename MaskType=grb::NoMask>
    auto multiply(AMatrixType const &A, BMatrixType const &B,
                  SemiringType semiring_op, MaskType const &Mask=MaskType());  // const& ?
\end{verbatim}

%-----------------------------------------------------------------------------
\section{{\sf mxv}: matrix-vector multiply}

\paragraph{\syntax}

\begin{minted}{c++}
    // overload using a semiring
    template<typename WVectorType,
             typename MaskType,
             typename AccumulatorType,
             typename SemiringType,
             typename AMatrixType,
             typename BMatrixType>
    void mxv(WVectorType            &w,
             MaskType         const &mask,
             AccumulatorType         accum,   // pass by value or const&?
             SemiringType            op,      // pass by value or const&?
             AMatrixType      const &A,
             UVectorType      const &u,
             OutputControlEnum       outp = MERGE);  //or bool replace_flag = false);

    // ...or...overload using two binary operators
    template<typename WVectorType,
             typename MaskType,
             typename AccumulatorType,
             typename ReduceType,
             typename MapType,
             typename AMatrixType,
             typename UVectorType>
    void mxv(WVectorType            &w,
             MaskType         const &mask,
             AccumulatorType         accum,   // pass by value or const&?
             ReduceType              reduce,  // pass by value or const&?
             MapType                 map,     // pass by value or const&?
             AMatrixType      const &A,
             UVectorType      const &u,
             OutputControlEnum       outp = MERGE);  //or bool replace_flag = false);
\end{minted}

%-----------------------------------------------------------------------------
\section{{\sf vxm}: vector-matrix multiply}

\paragraph{\syntax}

\begin{minted}{c++}
    // overload using a semiring
    template<typename WVectorType,
             typename MaskType,
             typename AccumulatorType,
             typename SemiringType,
             typename UVectorType,
             typename AMatrixType>
    void vxm(WVectorType            &w,
             MaskType         const &mask,
             AccumulatorType         accum,   // pass by value or const&?
             SemiringType            op,      // pass by value or const&?
             UVectorType      const &u,
             AMatrixType      const &A,
             OutputControlEnum       outp = MERGE);  //or bool replace_flag = false);

    // ...or...overload using two binary operators
    template<typename WVectorType,
             typename MaskType,
             typename AccumulatorType,
             typename ReduceType,
             typename MapType,
             typename UVectorType,
             typename AMatrixType>
    void vxm(WVectorType            &w,
             MaskType         const &mask,
             AccumulatorType         accum,   // pass by value or const&?
             ReduceType              reduce,  // pass by value or const&?
             MapType                 map,     // pass by value or const&?
             UVectorType      const &u,
             AMatrixType      const &A,
             OutputControlEnum       outp = MERGE);  //or bool replace_flag = false);
\end{minted}


%-----------------------------------------------------------------------------
\section{{\sf ewisemult}: element-wise multiplication, set intersection}

\paragraph{\syntax}

\begin{minted}{c++}
    // grb::vector overload
    template<typename T, typename I, typename Hint, typename Allocator,
             typename MaskType,
             typename AccumType,
             typename BinaryOpType,
             typename UVectorType,
             typename VVectorType>
    void ewisemult(vector<T, I, Hint, Allocator>     &w,
                   MaskType                    const &mask,
                   AccumType                   const &accum,
                   BinaryOpType                       op,
                   UVectorType                 const &u,
                   VVectorType                 const &v,
                   OutputControlEnum                  outp = MERGE);  // default val?

    // grb::matrix overload
    template<typename T, typename I, typename Hint, typename Allocator,
             typename MaskType,
             typename AccumType,
             typename BinaryOpType,
             typename AMatrixType,
             typename BMatrixType>
    void ewisemult(matrix<T, I, Hint, Allocator>     &C,
                   MaskT                       const &Mask,
                   AccumT                      const &accum,
                   BinaryOpT                          op,
                   AMatrixT                    const &A,
                   BMatrixT                    const &B,
                   OutputControlEnum                  outp = MERGE);  // default val?
\end{minted}


%-----------------------------------------------------------------------------
\section{{\sf ewiseadd}: element-wise addition}

\paragraph{\syntax}

\begin{minted}{c++}
    // grb::vector overload
    template<typename T, typename I, typename Hint, typename Allocator,
             typename MaskType,
             typename AccumType,
             typename BinaryOpType,
             typename UVectorType,
             typename VVectorType>
    void ewiseadd(vector<T, I, Hint, Allocator>     &w,
                  MaskType                    const &mask,
                  AccumType                   const &accum,
                  BinaryOpType                       op,
                  UVectorType                 const &u,
                  VVectorType                 const &v,
                  OutputControlEnum                  outp = MERGE);  // default val?

    // grb::matrix overload
    template<typename T, typename I, typename Hint, typename Allocator,
             typename MaskType,
             typename AccumType,
             typename BinaryOpType,
             typename AMatrixType,
             typename BMatrixType>
    void ewiseadd(matrix<T, I, Hint, Allocator>     &C,
                  MaskT                       const &Mask,
                  AccumT                      const &accum,
                  BinaryOpT                          op,
                  AMatrixT                    const &A,
                  BMatrixT                    const &B,
                  OutputControlEnum                  outp = MERGE);  // default val?
\end{minted}


%-----------------------------------------------------------------------------
%-----------------------------------------------------------------------------
\subsection{{\sf extract}: }

%-----------------------------------------------------------------------------
\subsection{{\sf extract}: standard vector variant}

\paragraph{\syntax}

\begin{minted}{c++}
    // standard grb::vector variant
    template<typename WVectorType,     // typename T, typename I, typename Hint, typename Allocator,
             typename MaskType,
             typename AccumType,
             typename UVectorType,
             typename SequenceType>
    void extract(WVectorType             &w,  // should we use vector<T, I, Hint, Allocator>?
                 MaskType          const &mask,
                 AccumType         const &accum,
                 UVectorType       const &u,
                 SequenceType      const &indices,
                 OutputControlEnum        outp = MERGE);  // default val?
\end{minted}

%-----------------------------------------------------------------------------
\subsection{{\sf extract}: standard matrix variant}

\paragraph{\syntax}

\begin{minted}{c++}
    // standard grb::matrix variant
    template<typename T, typename I, typename Hint, typename Allocator,
             typename MaskType,
             typename AccumType,
             typename AMatrixType,
             typename RowSequenceType,
             typename ColSequenceType>
    void extract(matrix<T, I, Hint, Allocator>  &C,
                 MaskT                    const &mask,
                 AccumT                   const &accum,
                 AMatrixT                 const &A,
                 RowSequenceT             const &row_indices,
                 ColSequenceT             const &col_indices,
                 OutputControlEnum               outp = MERGE);  // default val?
\end{minted}

%-----------------------------------------------------------------------------
\subsection{{\sf extract}: column (row) variant}

\paragraph{\syntax}

\begin{minted}{c++}
    // standard grb::matrix variant
    template<typename T, typename I, typename Hint, typename Allocator,
             typename MaskType,
             typename AccumType,
             typename AMatrixType,
             typename RowSequenceType>
    void extract(vector<T, I, Hint, Allocator>  &w,
                 MaskT                    const &mask,
                 AccumT                   const &accum,
                 AMatrixT                 const &A,
                 RowSequenceT             const &row_indices,
                 index_t                         col_index,
                 OutputControlEnum               outp = MERGE);  // default val?
\end{minted}


%-----------------------------------------------------------------------------
\section{{\sf assign}: }

%-----------------------------------------------------------------------------
\subsection{{\sf assign}: standard vector variant}

\paragraph{\syntax}

\begin{minted}{c++}
    template<typename T, typename I, typename Hint, typename Allocator,
             typename MaskT,
             typename AccumT,
             typename UVectorT,
             typename SequenceT,
             typename std::enable_if_t<is_vector_v<UVectorT>, int> = 0,
             typename ...WTags>
    void assign(vector<T, I, Hint, Allocator>      &w,
                MaskT                    const  &mask,
                AccumT                   const  &accum,
                UVectorT                 const  &u,
                SequenceT                const  &indices,
                OutputControlEnum                outp = MERGE)
\end{minted}

%-----------------------------------------------------------------------------
\subsection{{\sf assign}: standard matrix variant}

\paragraph{\syntax}

\begin{minted}{c++}
    template<typename CMatrixT, // it isn't enough to do T, I, Hint, Allocator
             typename MaskT,
             typename AccumT,
             typename AMatrixT,
             typename RowSequenceT,
             typename ColSequenceT,
             typename std::enable_if_t<is_matrix_v<AMatrixT>, int> = 0>
    inline void assign(CMatrixT              &C,
                       MaskT           const &Mask,
                       AccumT          const &accum,
                       AMatrixT        const &A,
                       RowSequenceT    const &row_indices,
                       ColSequenceT    const &col_indices,
                       OutputControlEnum      outp = MERGE)
\end{minted}


%-----------------------------------------------------------------------------
\subsection{{\sf assign}: column variant}

\paragraph{\syntax}

\begin{minted}{c++}
    template<typename CScalarT,
             typename MaskT,
             typename AccumT,
             typename UVectorT,
             typename SequenceT,
             typename ...CTags>
    inline void assign(Matrix<CScalarT, CTags...>  &C,
                       MaskT                 const &mask,  // a vector
                       AccumT                const &accum,
                       UVectorT              const &u,
                       SequenceT             const &row_indices,
                       IndexType                    col_index,
                       OutputControlEnum            outp = MERGE)
\end{minted}


%-----------------------------------------------------------------------------
\subsection{{\sf assign}: row variant}

\paragraph{\syntax}

\begin{minted}{c++}
    template<typename CScalarT,
             typename MaskT,
             typename AccumT,
             typename UVectorT,
             typename SequenceT,
             typename ...CTags>
    inline void assign(Matrix<CScalarT, CTags...>  &C,
                       MaskT                 const &mask,  // a vector
                       AccumT                const &accum,
                       UVectorT              const &u,
                       IndexType                    row_index,
                       SequenceT             const &col_indices,
                       OutputControlEnum            outp = MERGE)
\end{minted}


%-----------------------------------------------------------------------------
\subsection{{\sf assign}: vector constant variant}

\paragraph{\syntax}

\begin{minted}{c++}
    template<typename WVectorT,
             typename MaskT,
             typename AccumT,
             typename ValueT,
             typename SequenceT,
             typename std::enable_if<
                 std::is_convertible<ValueT,
                                     typename WVectorT::ScalarType>::value,
                 int>::type = 0>
    inline void assign(WVectorT                     &w,
                       MaskT                const   &mask,
                       AccumT               const   &accum,
                       ValueT                        val,
                       SequenceT            const   &indices,
                       OutputControlEnum             outp = MERGE)
\end{minted}


%-----------------------------------------------------------------------------
\subsection{{\sf assign}: matrix constant variant}

\paragraph{\syntax}

\begin{minted}{c++}
    template<typename CMatrixT,
             typename MaskT,
             typename AccumT,
             typename ValueT,
             typename RowSequenceT,
             typename ColSequenceT,
             typename std::enable_if<
                 std::is_convertible<ValueT,
                                     typename CMatrixT::ScalarType>::value,
                 int>::type = 0>
    inline void assign(CMatrixT             &C,
                       MaskT          const &Mask,
                       AccumT         const &accum,
                       ValueT                val,
                       RowSequenceT   const &row_indices,
                       ColSequenceT   const &col_indices,
                       OutputControlEnum     outp = MERGE)
\end{minted}



%-----------------------------------------------------------------------------
%-----------------------------------------------------------------------------
\subsection{{\sf apply}: }

\paragraph{\syntax}

\begin{minted}{c++}

\end{minted}


%-----------------------------------------------------------------------------
\subsection{{\sf select}: }

\paragraph{\syntax}

\begin{minted}{c++}

\end{minted}


%-----------------------------------------------------------------------------
\subsection{{\sf reduce}: }

\paragraph{\syntax}

\begin{minted}{c++}

\end{minted}


%-----------------------------------------------------------------------------
\subsection{{\sf transpose}: }

\paragraph{\syntax}

\begin{minted}{c++}

\end{minted}


%-----------------------------------------------------------------------------
\subsection{{\sf kronecker}: }

\paragraph{\syntax}

\begin{minted}{c++}

\end{minted}


%-----------------------------------------------------------------------------
%-----------------------------------------------------------------------------
