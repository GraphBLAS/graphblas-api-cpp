\chapter{Classes and Methods}
\label{Chp:Methods}

This chapter defines the behavior of all the methods in the GraphBLAS C API.
All methods can be declared for use in programs by including the {\tt GraphBLAS.h} header file.

We would like to emphasize that no GraphBLAS method will imply a predefined order over any associative operators. Implementations of the GraphBLAS are encouraged to exploit associativity to optimize performance of any GraphBLAS method. This holds even if the definition of the GraphBLAS method implies a fixed order for the associative operations.

%-----------------------------------------------------------------------------
\section{Context Methods}

%-----------------------------------------------------------------------------
\section{Object Classes and Methods}

This section describes methods that setup and operate on GraphBLAS opaque objects
but are not part of the the GraphBLAS math specification.  \scott{Does this need to be said?  Is it accurate?}

%-----------------------------------------------------------------------------
\subsection{Algebra Classes}

\scott{There are no separate "Types".  UnaryOp and BinaryOps are all callable objects.  Monoids and Semirings are class templates}

We could further define the concept and give a few examples of functions, functors and lambdas  that can be used where a unaryop, binaryop, monoid and semiring are needed.

Do we provide and list all of the predefined structs?

subsubsection{


%-----------------------------------------------------------------------------
\subsection{Vector Class}

This section defines the minimum required public interface to the "API" (I refer to it
as the frontend) Vector class.  \scott{Are implementers free to add to the public API?
I think it is possible it may be necessary?}

\subsubsection{Template and Type Definitions}

No subclassing in API

\begin{verbatim}
    template<typename ScalarT, typename IndexT, typename... TagsT>
    class Vector
    {
    public:
        typedef ScalarT ScalarType;
        typedef IndexT  IndexType;

        ...
    };
\end{verbatim}

We need to defined the template parameters:
\begin{itemize}[leftmargin=1.1in]
\item[\sf ScalarT]  The type of the elements.  Must they meet the requirements of CopyAssignable and CopyConstructible (pre C++11) or something else.  \scott{Here is the statement from the C++ Standard Graph Library proposal: "The graph value type defined by the user.  It can be most valid C++ value type including class, struct, tuple, union, enum, array, reference or scalar value. If no value is needed then the empty\_value struct can be used."}
\item[\sf IndexT] this type is used for a vertex’s index. \scott{Should we restrict to unsigned integer types?}
\item[\sf TagsT] variadic arguments TBD
\end{itemize}

What are the possible types sent through TagsT?

The class needs to define a few typedefs
\begin{itemize}[leftmargin=1.1in]
\item[\sf ScalarType] Echos the ScalarT template argument used
\item[\sf IndexType]  Echos the IndexT template argument used
\end{itemize}

\scott{class is a base class}
\scott{default constructor deleted}
\scott{destructor not virtual, do we allow implicit destructor?}
\scott{"dup" functionality: copy ctor and assignment operator both? What form?}
\scott{Do we require move semantics}


\subsubsection{Standard Constructors}

\paragraph{\syntax}

\begin{verbatim}
    Vector() = delete;         // default construction not supported
    Vector(Vector const &rsh); // Should we specify copy constructor?
    Vector(Vector &&rhs);      // Should we specify move constructor?
\end{verbatim}

\paragraph{Description}

\scott{How should we document the various constructors}

Default construction is not supported; i.e., a size must be specified.

Copying and moving vectors of exactly the same type could be supported.  
Copy would be O(nsize) complexity, move could be O(1).


\subsubsection{Constructor: create a new Vector}

\paragraph{\syntax}

\begin{verbatim}
    Vector(IndexType nsize);   // Standard constructor
\end{verbatim}

\paragraph{Parameters}

\begin{itemize}[leftmargin=1.1in]
    \item[{\sf nsize}] ({\sf IN}) The initial size of the vector being created.
\end{itemize}

\paragraph{Exceptions}

\begin{itemize}[leftmargin=2.1in]
    \item[{\sf grb::PanicException}]   Unknown internal error.
    \item[{\sf std::bad\_alloc}]        Not enough memory available for operation.
    \item[{\sf std::invalid\_argument}] {\sf nsize} is zero.
\end{itemize}

\paragraph{Description}

Creates a new vector $\vector{v}$ that can store {\sf ScalarT} type elements, with size {\sf nsize}, 
and empty $\mathbf{L}(\vector{v})$.

\scott{should we leverage as many STL exceptions classes as possible?}

%-----------------------
\subsubsection{Destructor}

\paragraph{\syntax}

\begin{verbatim}
    ~Vector();                  // not virtual (i.e. no subclassing)
\end{verbatim}

\paragraph{Description}

Releases all resources held by this container including all implementation data structures.
The destructor is not virtual as this class is not intended to participate in runtime polymorphism.

%-----------------------
\subsubsection{Assignment Operator}

\paragraph{\syntax}

\begin{verbatim}
    Vector& operator=(Vector const &rhs)
\end{verbatim}

\paragraph{Parameters}

\begin{itemize}[leftmargin=1.1in]
    \item[{\sf rhs}] ({\sf IN}) The vector to copy from (was constructed with the same template arguments).
\end{itemize}

\paragraph{Exceptions}

\begin{itemize}[leftmargin=2.1in]
    \item[{\sf grb::PanicException}]   Unknown internal error.
    \item[{\sf std::bad\_alloc}]        Not enough memory available for operation.
\end{itemize}

\paragraph{Description}

A copy is made of {\sf rsh} into this vector.  This vector changes size and content to match {\sf rhs}.
On return the state of {\sf rhs} is unchanged.

\scott{Should we support copy assignment?}  


%-----------------------
\subsubsection{Move operator}

\paragraph{\syntax}

\begin{verbatim}
    Vector& operator=(Vector const &&rhs)
\end{verbatim}

\paragraph{Exceptions}

\begin{itemize}[leftmargin=2.1in]
    \item[{\sf grb::PanicException}]   Unknown internal error.
\end{itemize}

\paragraph{Description}

The contents of {\sf rhs} moved into this vector.  This vector changes size and content to match {\sf rhs}.
On return the state of {\sf rhs} is "indeterminate".

\scott{Should we support move assignment?}
\scott{is this noexcept?}

%-----------------------
\subsubsection{{\sf build} Method: from iterators}

\scott{How many different overloads should be supported?}

\paragraph{\syntax}

\begin{verbatim}
    // build methods
    template<typename RAIteratorI,
             typename RAIteratorV,
             typename BinaryOpT = GraphBLAS::Second<ScalarT> >
    void build(RAIteratorI  i_it,
               RAIteratorV  v_it,
               IndexT       nvals,   // size_t?
               BinaryOpT    dup = BinaryOpT());
\end{verbatim}

\paragraph{Template Parameters}

\begin{itemize}[leftmargin=1.1in]
    \item[{\sf RAIteratorI}] A random access iterator over IndexT objects.  (type traits needed to enforce?)
    \item[{\sf RAIteratorV}] A random access iterator over ScalarT objects.  (type traits needed to enforce?)
    \item[{\sf BinarOpT}]    A callable in the form of a binary function
                             that takes two ScalarT items and returns a ScalarT. (type traits needed to enforce)
\end{itemize}

\paragraph{Parameters}

\begin{itemize}[leftmargin=1.1in]
    \item[{\sf i\_it}]   ({\sf IN}) OUT? Iterator over a sequence of indices
    \item[{\sf v\_it}]   ({\sf IN}) OUT? Iterator over a sequence of values
    \item[{\sf nvals}]   ({\sf IN}) The number of entries that will be iterated over from each sequence
    \item[{\sf dup}]     ({\sf IN}) An associative and commutative binary operator
                                    to apply when duplicate values for the same
                                    location (index) are present in the input sequences.
\end{itemize}

\paragraph{Exceptions}

\begin{itemize}[leftmargin=2.1in]
    \item[{\sf grb::PanicException}]   Unknown internal error.
    \item[{\sf std::bad\_alloc}]       Not enough memory available for operation.
    \item[{\sf std::out\_of\_range}]   A value in the index sequence is outside the
                                       allowed range for the vector.
    \item[{\sf grb::OutputNotEmpty}]   Vector is not empty when this is called.                                  
\end{itemize}

\paragraph{Description}

%-----------------------
\subsubsection{{\sf build} Method: from containers.}

\scott{How many different overloads should be supported?}

\paragraph{\syntax}

\begin{verbatim}
    template<typename BinaryOpT = GraphBLAS::Second<ScalarType> >
    inline void build(std::vector<IndexT>     const &indices,
                      std::vector<ScalarT>    const &values,
                      BinaryOpT                      dup = BinaryOpT());
\end{verbatim}

\paragraph{Template Parameters}

\begin{itemize}[leftmargin=1.1in]
    \item[{\sf BinarOpT}]    A callable in the form of a binary function
                             that takes two ScalarT items and returns a ScalarT. (type traits needed to enforce)
\end{itemize}

\paragraph{Parameters}

\begin{itemize}[leftmargin=1.1in]
    \item[{\sf indices}]  ({\sf IN}) container of indices
    \item[{\sf values}]   ({\sf IN}) container of values
    \item[{\sf dup}]      ({\sf IN}) An associative and commutative binary operator
                                     to apply when duplicate values for the same
                                     location (index) are present in the input sequences.
\end{itemize}

\paragraph{Exceptions}

\begin{itemize}[leftmargin=2.1in]
    \item[{\sf grb::PanicException}]   Unknown internal error.
    \item[{\sf grb::BadSize}]          indices and values containers do not have
                                       the same number of elements (or should we 
                                       also provide nvals? In which case, BadSize 
                                       means not enough values in containers).
    \item[{\sf std::bad\_alloc}]       Not enough memory available for operation.
    \item[{\sf std::out\_of\_range}]   A value in the index sequence is outside the
                                       allowed range for the vector.
    \item[{\sf grb::OutputNotEmpty}]   Vector is not empty when this is called.                                  
\end{itemize}

\paragraph{Description}

%-----------------------
\subsubsection{{\sf clear} Method}
\begin{verbatim}
    void clear(); // noexcept?
\end{verbatim}

%-----------------------
\subsubsection{{\sf size} Method}
\begin{verbatim}
    IndexType size() const noexcept;
\end{verbatim}

%-----------------------
\subsubsection{{\sf nvals} Method}
\begin{verbatim}
    IndexType nvals() const noexcept;
\end{verbatim}

%-----------------------
\subsubsection{{\sf resize} Method}
\begin{verbatim}
    void resize(IndexType new_size);
\end{verbatim}


%-----------------------
\subsubsection{{\sf hasElement} Method}
\begin{verbatim}
    bool hasElement(IndexType index) const;
\end{verbatim}

%-----------------------
\subsubsection{{\sf setElement} Method}
\begin{verbatim}
    void setElement(IndexType index, ScalarT const &new_val);
\end{verbatim}

%-----------------------
\subsubsection{{\sf removeElement} Method}
\begin{verbatim}
    void removeElement(IndexType index);
\end{verbatim}

%-----------------------
\subsubsection{{\sf extractElement} Method}
\begin{verbatim}
    ScalarT extractElement(IndexType index) const;
\end{verbatim}


%-----------------------
\subsubsection{{\sf extractTuples} Method}
\begin{verbatim}
    template<typename RAIteratorIT,
             typename RAIteratorVT>
    void extractTuples(RAIteratorIT        i_it,
                       RAIteratorVT        v_it) const;

    void extractTuples(IndexArrayType        &indices,
                       std::vector<ScalarT>  &values) const;
\end{verbatim}

%-----------------------
\subsubsection{Iterators or ranges?}

\scott{Do we require iterators to iterate over stored elements?}


%-----------------------------------------------------------------------------
%-----------------------------------------------------------------------------
\subsection{Matrix Class}

This section defines the minimum required public interface to the "API" (I refer to it
as the frontend) Matrix class.  \scott{Are implementers free to add to the public API?
I think it is possible it may be necessary?}

%-----------------------------------------------------------------------------
\subsection{Descriptor Class...There isn't one}

\scott{There should be no Descriptors...I hope...maybe Views}.

%-----------------------------------------------------------------------------
%-----------------------------------------------------------------------------
\subsection{Views}

\subsubsection{transpose method, TransposeView class}

Only for matrices.  Do we need to specify the TransposeView class?
Is the class an implementation detail?  Can we specify the method without stating its return type?


\subsubsection{structure method, StructureView class}

Only for masks.  Do we need to specify the StructureView class?
Is the class an implementation detail?  Can we specify the method without stating its return type?


\subsubsection{complement method, ComplementView class}

Only for masks.  Do we need to specify the ComplementView class?
Is the class an implementation detail?  Can we specify the method without stating its return type?


\subsubsection{Composing mask views}

Only complement(structure(mask)) is valid  (structure(complement(mask) is not valid).

Do we talk about StructuralComplementView class?

Is the class an implementation detail?  Can we specify the method without stating its return type?

%-----------------------------------------------------------------------------
\subsection{{\sf free} Methods}

\scott{should be replaced with class destructors}.

%-----------------------------------------------------------------------------
\subsection{{\sf wait} Methods}

\scott{should be moved to a class method in each object type.}

%-----------------------------------------------------------------------------
\subsection{{\sf error} Methods}

\scott{should be moved to a class method in each object type.}

%-----------------------------------------------------------------------------
%-----------------------------------------------------------------------------
%-----------------------------------------------------------------------------
\section{GraphBLAS Operations}

%-----------------------------------------------------------------------------
\subsection{{\sf mxm}: Matrix-matrix multiply}

Multiplies a matrix with another matrix on a semiring. The result is a matrix.

\paragraph{\syntax}

\begin{verbatim}
        GrB_Info GrB_mxm(GrB_Matrix             C,
                         const GrB_Matrix       Mask,
                         const GrB_BinaryOp     accum,
                         const GrB_Semiring     op,
                         const GrB_Matrix       A,
                         const GrB_Matrix       B,
                         const GrB_Descriptor   desc);
\end{verbatim}

\paragraph{Parameters}

\begin{itemize}[leftmargin=1.1in]
    \item[{\sf C}]    ({\sf INOUT}) An existing GraphBLAS matrix. On input,
    the matrix provides values that may be accumulated with the result of the
    matrix product.  On output, the matrix holds the results of the
    operation.

    \item[{\sf Mask}] ({\sf IN}) An optional ``write'' mask that controls which
    results from this operation are stored into the output matrix {\sf C}. The 
    mask dimensions must match those of the matrix {\sf C}. If the 
    {\sf GrB\_STRUCTURE} descriptor is {\em not} set for the mask, the domain 
    of the {\sf Mask} matrix must be of type {\sf bool} or any of the predefined 
    ``built-in'' types in Table~\ref{Tab:PredefinedTypes}.  If the default
    mask is desired (\ie, a mask that is all {\sf true} with the dimensions of {\sf C}), 
    {\sf GrB\_NULL} should be specified. \scott{STRUCTURE\_ONLY changes.}

    \item[{\sf accum}] ({\sf IN}) An optional binary operator used for accumulating
    entries into existing {\sf C} entries.
    %: ${\sf accum} = \langle \bDout({\sf accum}),\bDin1({\sf accum}),
    %\bDin2({\sf accum}), \odot \rangle$. 
    If assignment rather than accumulation is
    desired, {\sf GrB\_NULL} should be specified.

    \item[{\sf op}]   ({\sf IN}) The semiring used in the matrix-matrix
    multiply.
    %: ${\sf op}=\langle \bDout({\sf op}),\bDin1({\sf op}),\bDin2({\sf op}),\oplus,\otimes,0 \rangle$.

    \item[{\sf A}]    ({\sf IN}) The GraphBLAS matrix holding the values
    for the left-hand matrix in the multiplication.

    \item[{\sf B}]    ({\sf IN}) The GraphBLAS matrix holding the values for
    the right-hand matrix in the multiplication.

    \item[{\sf desc}] ({\sf IN}) An optional operation descriptor. If
    a \emph{default} descriptor is desired, {\sf GrB\_NULL} should be
    specified. Non-default field/value pairs are listed as follows:  \\
    
    \hspace*{-2em}\begin{tabular}{lllp{2.7in}}
        Param & Field  & Value & Description \\
        \hline
        {\sf C}    & {\sf GrB\_OUTP} & {\sf GrB\_REPLACE} & Output matrix {\sf C}
        is cleared (all elements removed) before the result is stored in it.\\

        {\sf Mask} & {\sf GrB\_MASK} & {\sf GrB\_STRUCTURE}   & The write mask is
        constructed from the structure (pattern of stored values) of the input
        {\sf Mask} matrix. The stored values are not examined.\\

        {\sf Mask} & {\sf GrB\_MASK} & {\sf GrB\_COMP}   & Use the
        complement of {\sf Mask}. \\

        {\sf A}    & {\sf GrB\_INP0} & {\sf GrB\_TRAN}   & Use transpose of {\sf A}
        for the operation. \\

        {\sf B}    & {\sf GrB\_INP1} & {\sf GrB\_TRAN}   & Use transpose of {\sf B}
        for the operation. \\
    \end{tabular}
\end{itemize}

\paragraph{Return Values}

\begin{itemize}[leftmargin=2.1in]
    \item[{\sf GrB\_SUCCESS}]         In blocking mode, the operation completed
    successfully. In non-blocking mode, this indicates that the compatibility 
    tests on dimensions and domains for the input arguments passed successfully. 
    Either way, output matrix {\sf C} is ready to be used in the next method of
    the sequence.

    \item[{\sf GrB\_PANIC}]           Unknown internal error.

    \item[{\sf GrB\_INVALID\_OBJECT}] This is returned in any execution mode 
    whenever one of the opaque GraphBLAS objects (input or output) is in an invalid 
    state caused by a previous execution error.  Call {\sf GrB\_error()} to access 
    any error messages generated by the implementation.

    \item[{\sf GrB\_OUT\_OF\_MEMORY}] Not enough memory available for the operation.

    \item[{\sf GrB\_UNINITIALIZED\_OBJECT}] One or more of the GraphBLAS objects 
    has not been initialized by a call to {\sf new} (or {\sf Matrix\_dup} for matrix
    parameters).

    \item[{\sf GrB\_DIMENSION\_MISMATCH}] Mask and/or matrix
    dimensions are incompatible.

    \item[{\sf GrB\_DOMAIN\_MISMATCH}]    The domains of the various matrices are
    incompatible with the corresponding domains of the semiring or
    accumulation operator, or the mask's domain is not compatible with {\sf bool}
    (in the case where {\sf desc[GrB\_MASK].GrB\_STRUCTURE} is not set).\scott{STRUCTURE\_ONLY changes.}
\end{itemize}

\paragraph{Description}

%-----------------------------------------------------------------------------
\subsection{{\sf mxv}: Matrix-vector multiply}

\subsection{{\sf vxm}: Vector-matrix multiply}


%-----------------------------------------------------------------------------
\subsection{ops\_ewisemult\_ewiseadd}

%-----------------------------------------------------------------------------
\subsection{ops\_extract}

%-----------------------------------------------------------------------------
\subsection{ops\_assign} 

%-----------------------------------------------------------------------------
\subsection{ops\_apply}

%-----------------------------------------------------------------------------
\subsection{ops\_reduce\_transpose}

%-----------------------------------------------------------------------------
\subsection{ops\_kronecker}

%-----------------------------------------------------------------------------
%-----------------------------------------------------------------------------
